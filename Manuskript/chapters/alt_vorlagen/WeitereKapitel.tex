\chapter{Weitere Kapitel}

Die Gliederung hängt natürlich vom Thema und von der Lösungsstrategie ab. Als nützliche
Anhaltspunkte können die Entwicklungsstufen oder - schritte z.B. der Softwareentwicklung betrachtet werden. Nützliche Gesichtspunkte erhält und erkennt man, wenn man sich
\begin{itemize}
  \item in die Rolle des Lesers oder
  \item in die Rolle des Entwicklers, der die Arbeit z.B. fortsetzen, ergänzen oder pflegen soll,
\end{itemize}
versetzt. In der Regel wird vorausgesetzt, dass die Leser einen fachlichen Hintergrund haben - z.B. Informatik studiert haben. D.h. nur in besonderen, abgesprochenen Fällen schreibt man in populärer Sprache, so dass auch Nicht-Fachleute die Ausarbeitung prinzipiell lesen und verstehen können.

Die äußere Gestaltung der Ausarbeitung hinsichtlich Abschnittformate, Abbildungen, mathematische Formeln usw. wird in \hyperref[Stile]{Kapitel~\ref*{Stile}} kurz dargestellt.