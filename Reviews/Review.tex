\documentclass[a4paper,12pt]{book}

\usepackage[a4paper, inner=1.7cm, outer=2.7cm, top=1.5cm, bottom=2cm, bindingoffset=1.2cm]{geometry}
\usepackage[german]{babel}
\usepackage{microtype}
\usepackage{fancyhdr}


\pagestyle{fancy}
\fancyfoot{}
\fancyfoot[C]{\thepage}

\begin{document}

%Bitte noch Gruppennummer eintragen
\chapter*{Review zur Ausarbeitung der Gruppe 7:}

\section*{Allgemeine Bewertung / Allgemeiner Eindruck}
Insgesamt sind die einzelnen Themen gut und ausführlich beschrieben, allerdings fallen häufige Grammatik- und Satzbaufehler ins Auge; insbesondere wenn aus englischsprachigen Quellen ins Deutsche übersetzt wurde.\\
Weiterhin sind manche Abschnitte sehr repetitiv konstruiert und sollten überarbeitet werden (siehe einzelne Anmerkungen).
Auch inhaltlich wird sich bei Formulierungen wiederholt auf Kontext von drei Sätzen vorher bezogen, was den Lesefluss an manchen Stellen beeinträchtigt.

\section*{Anmerkungen}

\textbf{Anmerkung 1 (Seite 2, Zeile 6):}
"Platzkomplexität" $\rightarrow$ "Raumkomplexität"\\
\\
%\textbf{Stellungnahme zu 1:}
%Lorem ipsum dolor sit amet, consectetuer adipiscing elit. Aenean commodo ligula eget dolor. Vivamus elementum semper nisi. Aenean vulputate eleifend tellus. Aenean leo ligula.\\
%\\

\noindent
\textbf{Anmerkung 2 (Seite 2, Zeile 13):}
"geoinformation systems" $\rightarrow$ "geographic information systems"\\
\\
%\textbf{Stellungnahme zu 2:}
%Lorem ipsum dolor sit amet, consectetuer adipiscing elit. Aenean commodo ligula eget dolor. Vivamus elementum semper nisi. Aenean vulputate eleifend tellus. Aenean leo ligula.\\
%\\

\noindent
\textbf{Anmerkung 3 (Seite 5, Zeile 9):}
fehlendes Satzzeichen
\\ \\
\textbf{Stellungnahme zu 3:}
Satzzeichen hinzugefügt
\\

\noindent
\textbf{Anmerkung 4 (Seite 5, Zeile 10-11):}
Eventuell den Namen des Themas in Anführungszeichen setzen bzw. kursiv schreiben, um es besser hervorzuheben.\\
\\
\textbf{Stellungnahme zu 4:}
Das Thema ist nun in kursiv geschrieben
\\

\noindent
\textbf{Anmerkung 5 (Seite 5, Zeile 11):}
Kommasetzung beachten: [...] hat\emph{,} damals wie heute\emph{,} einen [...] \\
\\
\textbf{Stellungnahme zu 5:}
Kommasetzung korrigiert.\\

\noindent
\textbf{Anmerkung 6 (Seite 5, Zeile 12-14):}
Kommasetzung beachten, Satzbau ergibt sonst keinen Sinn. \\
\\
\textbf{Stellungnahme zu 6:}
Satzbau durch Doppelpunkt verbessert.
\\

\noindent
\textbf{Anmerkung 7 (Seite 5, Zeile 15):}
"sind Robotik (z.B. Pakete im Logistikbereich)" $\rightarrow$ sollte umformuliert werden; z.b. wie genau benutzen Pakete Pfadplanungsalgorithmen? \\
\\
\textbf{Stellungnahme zu 7:}
Textstelle um konkreteres Beispiel erweitert.
\\

\noindent
\textbf{Anmerkung 8 (Seite 5, Zeile 16-19):}
Zwei unabhängige Sätze sind besser als ein gigantischer. \\
\\
\textbf{Stellungnahme zu 8:}
Zu langen Satz in zwei Sätze geteilt.
\\

\noindent
\textbf{Anmerkung 9 (Seite 6, Punkt 2.2.1, Zeile 1):}
Zwei unabhängige Sätze sind besser. \\
\\
\textbf{Stellungnahme zu 9:}
Langen Satz in zwei Sätze unterteilt.
\\

\noindent
\textbf{Anmerkung 10 (Seite 6, Punkt 2.2.1, Zeile 3):}
"Wichtungen" $\rightarrow$ \emph{Ge}wichtungen \\
\\
\textbf{Stellungnahme zu 10:}
Wichtungen in \emph{Ge}wichtungen umbenannt.
\\

\noindent
\textbf{Anmerkung 11 (Seite 6, Punkt 2.2.1, Zeile 5):}
[...] wiederholt\emph{,} bis [...] \\
\\
\textbf{Stellungnahme zu 11:}
Satz umgestellt und Komma ergänzt.
\\

\noindent
\textbf{Anmerkung 12 (Seite 6, Punkt 2.2.1, Zeile 6-7):}
Die Breitensuche [...] \emph{ihre} Anwendung \\
\\
\textbf{Stellungnahme zu 12:}
\glqq seine \grqq zu \emph{ihre} korrigiert.
\\

\noindent
\textbf{Anmerkung 13 (Seite 7, Zeile 5):}
Fehlendes Leerzeichen nach Satzende. \\
\\
%\textbf{Stellungnahme zu 13:}
%Lorem ipsum dolor sit amet, consectetuer adipiscing elit. Aenean commodo ligula eget dolor. Vivamus elementum semper nisi. Aenean vulputate eleifend tellus. Aenean leo ligula.\\
%\\

\noindent
\textbf{Anmerkung 14 (Seite 6, Punkt 2.3.2, zweiter Abschnitt):}
Es wird von \emph{drei} Gruppen gesprochen, aber nur \emph{zwei} werden aufgezählt. \\
\\
%\textbf{Stellungnahme zu 14:}
%Lorem ipsum dolor sit amet, consectetuer adipiscing elit. Aenean commodo ligula eget dolor. Vivamus elementum semper nisi. Aenean vulputate eleifend tellus. Aenean leo ligula.\\
%\\

\noindent
\textbf{Anmerkung 15 (Seite 6, Punkt 2.3.2, zweiter Abschnitt, Zeile 8):}
"Markierung-knoten" sollte umformuliert werden. \\
\\
%\textbf{Stellungnahme zu 15:}
%Lorem ipsum dolor sit amet, consectetuer adipiscing elit. Aenean commodo ligula eget dolor. Vivamus elementum semper nisi. Aenean vulputate eleifend tellus. Aenean leo ligula.\\
%\\

\noindent
\textbf{Anmerkung 16 (Seite 8, Punkt 2.3.3, Zeile 1):}
[...] \emph{haben zitiert} $\rightarrow$ gibt es einen Grund, anstatt der Originalquelle das Zitat zu zitieren? \\
\\
%\textbf{Stellungnahme zu 16:}
%Lorem ipsum dolor sit amet, consectetuer adipiscing elit. Aenean commodo ligula eget dolor. Vivamus elementum semper nisi. Aenean vulputate eleifend tellus. Aenean leo ligula.\\
%\\

\noindent
\textbf{Anmerkung 17 (Seite 8):}
In Punkt 2.3.3 werden die Vorteile in einer Liste aufgezählt, in Punkt 2.3.4 in einem Fließtext. Dies sollte einheitlicher gestaltet werden. \\
\\
%\textbf{Stellungnahme zu 17:}
%Lorem ipsum dolor sit amet, consectetuer adipiscing elit. Aenean commodo ligula eget dolor. Vivamus elementum semper nisi. Aenean vulputate eleifend tellus. Aenean leo ligula.\\
%\\

\noindent
\textbf{Anmerkung 18 (Seite 8-9):}
Die Überschrift für den Pseudocode sollte auf derselben Seite stehen wie der Abschnitt selbst. \\
\\
%\textbf{Stellungnahme zu 18:}
%Lorem ipsum dolor sit amet, consectetuer adipiscing elit. Aenean commodo ligula eget dolor. Vivamus elementum semper nisi. Aenean vulputate eleifend tellus. Aenean leo ligula.\\
%\\

\noindent
\textbf{Anmerkung 19 (Seite 9):}
Schritt 2 und 3 sind als solche benannt, Schritt 1 nicht. Dies sollte einheitlich für alle Schritte sein. \\
\\
%\textbf{Stellungnahme zu 19:}
%Lorem ipsum dolor sit amet, consectetuer adipiscing elit. Aenean commodo ligula eget dolor. Vivamus elementum semper nisi. Aenean vulputate eleifend tellus. Aenean leo ligula.\\
%\\

\noindent
\textbf{Anmerkung 20 (Seite 10, Punkt 2.4.1, Zeile 1):}
Prof.ChitraBaggar $\rightarrow$ Leerzeichen einfügen. \\
\\
%\textbf{Stellungnahme zu 20:}
%Lorem ipsum dolor sit amet, consectetuer adipiscing elit. Aenean commodo ligula eget dolor. Vivamus elementum semper nisi. Aenean vulputate eleifend tellus. Aenean leo ligula.\\
%\\

\noindent
\textbf{Anmerkung 21 (Seite 10, Punkt 2.4.1, Zeile 1):}
Kommasetzung beachten: Vor das \emph{dass}, nicht dahinter. \\
\\
%\textbf{Stellungnahme zu 21:}
%Lorem ipsum dolor sit amet, consectetuer adipiscing elit. Aenean commodo ligula eget dolor. Vivamus elementum semper nisi. Aenean vulputate eleifend tellus. Aenean leo ligula.\\
%\\

\noindent
\textbf{Anmerkung 22 (Seite 10, Punkt 2.4.1, Zeile 1-3):}
Unsauberer Satzbau. Eventuell kein direktes Zitat verwenden, sondern umformulieren. \\
\\
%\textbf{Stellungnahme zu 22:}
%Lorem ipsum dolor sit amet, consectetuer adipiscing elit. Aenean commodo ligula eget dolor. Vivamus elementum semper nisi. Aenean vulputate eleifend tellus. Aenean leo ligula.\\
%\\

\noindent
\textbf{Anmerkung 23 (Seite 10, Punkt 2.4.2, Zeile 1):}
Was sind Gewichts-\emph{Nuss}-Zyklen? \\
\\
%\textbf{Stellungnahme zu 22:}
%Lorem ipsum dolor sit amet, consectetuer adipiscing elit. Aenean commodo ligula eget dolor. Vivamus elementum semper nisi. Aenean vulputate eleifend tellus. Aenean leo ligula.\\
%\\

\noindent
\textbf{Anmerkung 24 (Seite 10-11):}
Die Überschrift für den Pseudocode sollte auf derselben Seite stehen wie der Abschnitt selbst. \\
\\
%\textbf{Stellungnahme zu 24:}
%Lorem ipsum dolor sit amet, consectetuer adipiscing elit. Aenean commodo ligula eget dolor. Vivamus elementum semper nisi. Aenean vulputate eleifend tellus. Aenean leo ligula.\\
%\\

\noindent
\textbf{Anmerkung 25 (Seite 13, Abschnitt 2, Zeile 7):}
Heuristik\emph{,} immer $\rightarrow$ Komma entfernen. \\
\\
%\textbf{Stellungnahme zu 25:}
%Lorem ipsum dolor sit amet, consectetuer adipiscing elit. Aenean commodo ligula eget dolor. Vivamus elementum semper nisi. Aenean vulputate eleifend tellus. Aenean leo ligula.\\
%\\

\noindent
\textbf{Anmerkung 26 (Seite 13, Punkt 3.3, Zeile 1):}
Eine weiter $\rightarrow$ Eine \emph{weitere} \\
\\
%\textbf{Stellungnahme zu 26:}
%Lorem ipsum dolor sit amet, consectetuer adipiscing elit. Aenean commodo ligula eget dolor. Vivamus elementum semper nisi. Aenean vulputate eleifend tellus. Aenean leo ligula.\\
%\\

\noindent
\textbf{Anmerkung 27 (Seite 13, Punkt 3.3.1, Zeile 7 \& Seite 14, Zeile 1):}\\
Dreiecksungleich $\rightarrow$ Dreiecksungleich\emph{ung} \\
\\
%\textbf{Stellungnahme zu 27:}
%Lorem ipsum dolor sit amet, consectetuer adipiscing elit. Aenean commodo ligula eget dolor. Vivamus elementum semper nisi. Aenean vulputate eleifend tellus. Aenean leo ligula.\\
%\\

\noindent
\textbf{Anmerkung 28 (Seite 14, Punkt 3.3.2):}\\
"der Reach" $\rightarrow$ unsaubere Formulierung; warum nicht ins Deutsche übersetzen ("Reichweite")? (Außerdem wäre es \emph{die} Reach). \\
\\
%\textbf{Stellungnahme zu 28:}
%Lorem ipsum dolor sit amet, consectetuer adipiscing elit. Aenean commodo ligula eget dolor. Vivamus elementum semper nisi. Aenean vulputate eleifend tellus. Aenean leo ligula.\\
%\\

\noindent
\textbf{Anmerkung 29 (Seite 16, Punkt 4.1.1, Zeile 1):}\\
"Prof.Chitra Baggar" $\rightarrow$ fehlendes Leerzeichen; außerdem Vgl. Anmerkung 20: welche Version des Namens ist korrekt?\\
\\
%\textbf{Stellungnahme zu 29:}
%Lorem ipsum dolor sit amet, consectetuer adipiscing elit. Aenean commodo ligula eget dolor. Vivamus elementum semper nisi. Aenean vulputate eleifend tellus. Aenean leo ligula.\\
%\\

\noindent
\textbf{Anmerkung 30 (Seite 16, Punkt 4.1.2, Zeile 1):}\\
"dieser Dijkstra-Alogrithmus" $\rightarrow$ welcher? \\
\\
%\textbf{Stellungnahme zu 30:}
%Lorem ipsum dolor sit amet, consectetuer adipiscing elit. Aenean commodo ligula eget dolor. Vivamus elementum semper nisi. Aenean vulputate eleifend tellus. Aenean leo ligula.\\
%\\

\noindent
\textbf{Anmerkung 31 (Seite 17, Zeile 1):}\\
So haben [...] wie folgt beschrieben $\rightarrow$ unsauberer Satzbau. \\
\\
%\textbf{Stellungnahme zu 31:}
%Lorem ipsum dolor sit amet, consectetuer adipiscing elit. Aenean commodo ligula eget dolor. Vivamus elementum semper nisi. Aenean vulputate eleifend tellus. Aenean leo ligula.\\
%\\

\noindent
\textbf{Anmerkung 32 (Seite 18, Zeile 1):}\\
Kommasetzung beachten. \\
\\
%\textbf{Stellungnahme zu 32:}
%Lorem ipsum dolor sit amet, consectetuer adipiscing elit. Aenean commodo ligula eget dolor. Vivamus elementum semper nisi. Aenean vulputate eleifend tellus. Aenean leo ligula.\\
%\\

\noindent
\textbf{Anmerkung 33 (Seite 18, Zeile 1):}\\
"haben zitiert" $\rightarrow$ Warum wurde das Zitat zitiert? Vgl. Anmerkung 22 \\
\\
%\textbf{Stellungnahme zu 33:}
%Lorem ipsum dolor sit amet, consectetuer adipiscing elit. Aenean commodo ligula eget dolor. Vivamus elementum semper nisi. Aenean vulputate eleifend tellus. Aenean leo ligula.\\
%\\

\noindent
\textbf{Anmerkung 34 (Seite 17-19, Punkt 4):}\\
"die Autoren des Artikels" $\rightarrow$ Welcher Artikel? Dies wird nicht direkt erwähnt. \\
\\
%\textbf{Stellungnahme zu 34:}
%Lorem ipsum dolor sit amet, consectetuer adipiscing elit. Aenean commodo ligula eget dolor. Vivamus elementum semper nisi. Aenean vulputate eleifend tellus. Aenean leo ligula.\\
%\\

\noindent
\textbf{Anmerkung 35 (Seite 17-19, Punkt 4):}\\
"die Autoren des Artikels" tritt drei mal hintereinander auf. \\
\\
%\textbf{Stellungnahme zu 35:}
%Lorem ipsum dolor sit amet, consectetuer adipiscing elit. Aenean commodo ligula eget dolor. Vivamus elementum semper nisi. Aenean vulputate eleifend tellus. Aenean leo ligula.\\
%\\

\noindent
\textbf{Anmerkung 36 (Seite 20, Punkt 4.3, Zeile 4):}\\
"unter Anderem" wird großgeschrieben.\\
\\
\textbf{Stellungnahme zu 36:}
Satz umgestellt zu \emph{unter anderen Faktoren}\\

\noindent
\textbf{Anmerkung 37 (Seite 20, Punkt 4.3, Zeile 6):}\\
"Weg- findung und planung" $\rightarrow$ Wegfindung und -planung \\
\\
\textbf{Stellungnahme zu 37:}
Textstelle verbessert.
\\

\noindent
\textbf{Anmerkung 38 (Seite 20, Punkt 4.3, Zeile 12):}\\
Navigation in unübersichtlichen Umgebung" $\rightarrow$ Umgebung\emph{en} \\
\\
\textbf{Stellungnahme zu 38:}
Plural korrigiert.
\\

\noindent
\textbf{Anmerkung 39 (Seite 21, gesamter erster Abschnitt):}\\
Repetitiver Satzbau ("im Bereich", sowie "zu finden" steht zwei mal jeweils dicht beieinander). \\
\\
\textbf{Stellungnahme zu 39:}
Dopplungen aufgelöst.
\\

\noindent
\textbf{Anmerkung 40 (Seite 21, Zeile 9):}\\
Definition "Suche" erst ganz zum Schluss des Dokuments? 
\\ \\
\textbf{Stellungnahme zu 40:}
Definition  "Suche" in die Einleitung verschoben.
\\ \\

\noindent
\textbf{Anmerkung 41 (Seite 21, zweiter Abschnitt):}\\
"Breadth-first-Suche" $\rightarrow$ entweder komplett in Englisch, oder komplett in Deutsch, nicht beides vermischt. \\
\\
\textbf{Stellungnahme zu 41:}
Zu \glqq Breitensuche \grqq korrigiert.
\\

\noindent
\textbf{Anmerkung 42 (Seite 21, zweiter Abschnitt):}\\
"Der Greedy Dijkstra-Algorithmus der [...] vergeudet aber [...]" $\rightarrow$ Kommasetzung beachten. \\
\\
\textbf{Stellungnahme zu 42:}
Kommasetzung korrigiert.
\\

\noindent
\textbf{Anmerkung 43 (Seite 22, Zeile 3-5):}\\
"Best-First-Suchalgorithmus" vs. "Greedy best-first search" $\rightarrow$ einheitlich Deutsch oder Englisch verwenden. \\
\\
\textbf{Stellungnahme zu 43:}
Bezeichnungen vereinheitlicht.
\\

\noindent
\textbf{Anmerkung 44 (Seite 22, Zeile 5):}\\
"Der Greedy best-first search expandiert" $\rightarrow$ umgangssprachliche Formulierung. \\
\\
\textbf{Stellungnahme zu 44:}
\glqq-Algorithmus\grqq ergänzt.
\\

\noindent
\textbf{Anmerkung 45 (Seite 22, letzter Abschnitt, Zeile 2):}\\
Kommasetzung beachten: "[...] Anwendungsbereiche\emph{,} auf [...]" \\
\\
\textbf{Stellungnahme zu 45:}
Kommasetzung korrigiert.
\\

\noindent
\textbf{Anmerkung 46 (gesamtes Dokument):}
Quellenangaben stehen immer hinter dem Satzzeichen, was oftmals zu Verwirrung mit dem nächsten Satz führt. Daher besser die Quellenverweise immer \emph{vor} dem Satzzeichen einbinden. \\
\\
\textbf{Stellungnahme zu 46:}
Quellenangaben korrigiert, sodass diese vor dem Satzzeichen stehen.
\\

\end{document}