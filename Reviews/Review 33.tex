\documentclass[a4paper,12pt]{book}

\usepackage[a4paper, inner=1.7cm, outer=2.7cm, top=1.5cm, bottom=2cm, bindingoffset=1.2cm]{geometry}
\usepackage[german]{babel}
\usepackage{microtype}
\usepackage{fancyhdr}


\pagestyle{fancy}
\fancyfoot{}
\fancyfoot[C]{\thepage}

\begin{document}

%Bitte noch Gruppennummer eintragen
\chapter*{Review zur Ausarbeitung der Gruppe 7:}

\section*{Allgemeine Bewertung / Allgemeiner Eindruck}
Die starke Harmonie zwischen den Kapiteln ist mir aufgefallen. Die ausführliche Erklärung und Details haben das Thema richtig gefertigt.
Gliederung und die Anordnung den Kapiteln sind gut strukturiert. Aber die intensiven Informationen und Einzelheiten führen manchmal zu einer Verwirrung.
Es ist eine gute Idee, die Optimierungsmöglichkeiten ins Licht zu bringen, ich würde aber gerne über einen dritten Algorithmus lesen zu haben. 

\section*{Anmerkungen}

\textbf{Anmerkung 1 (Seite 1, Zeile 3):}
Wikipedia als Quelle würde ich nicht vertrauen.
\\
\\
%\textbf{Stellungnahme zu 1:}
%Lorem ipsum dolor sit amet, consectetuer adipiscing elit. Aenean commodo ligula eget dolor. Vivamus elementum semper nisi. Aenean vulputate eleifend tellus. Aenean leo ligula.\\
%\\
\textbf{Anmerkung 2 (Seite 3, 2.3.2 Prinzip): }
Der Begriff Topologie-Roadmap ist nicht definiert.
 \\
\\
%\textbf{Stellungnahme zu 2:}
%Lorem ipsum dolor sit amet, consectetuer adipiscing elit. Aenean commodo ligula eget dolor. Vivamus elementum semper nisi. Aenean vulputate eleifend tellus. Aenean leo ligula.\\
%\\
\textbf{Anmerkung 3 (Seite 4, Dijkstras Nachteile):}
Wird nicht klar definiert, was die beste Route ist.
\newline
Es wurde doch was davon implizit erwähnt \footnote{ 2.1  Was ist Pfadplanung? und 2.3.2 Prinzip}, aber war mir unklar, was genau für Eigenschaften der besten Route bei Dijkstra sind.
 \\
\\
%\textbf{Stellungnahme zu 3:}
%Lorem ipsum dolor sit amet, consectetuer adipiscing elit. Aenean commodo ligula eget dolor. Vivamus elementum semper nisi. Aenean vulputate eleifend tellus. Aenean leo ligula.\\
%\\
\textbf{Anmerkung 4 (alle Pseudo-Code Abbildungen):}
Man spart doch viel Aufwand bei einer einfachen Hinzufügung einer Abbildung, aber stattdessen wäre schöner, wenn der Pseudocode im Latex geschrieben wird. 
\\
\\
%\textbf{Stellungnahme zu 3:}
%Lorem ipsum dolor sit amet, consectetuer adipiscing elit. Aenean commodo ligula eget dolor. Vivamus elementum semper nisi. Aenean vulputate eleifend tellus. Aenean leo ligula.\\
%\\
\textbf{Anmerkung 5 (alle Abbildungen):}
Am besten jede Abbildung mit einer Referenz binden, sodass die Abbildung, und zwar beim Lesen, sich mit einem Klick schneller stellt.
%\textbf{Stellungnahme zu 3:}
%Lorem ipsum dolor sit amet, consectetuer adipiscing elit. Aenean commodo ligula eget dolor. Vivamus elementum semper nisi. Aenean vulputate eleifend tellus. Aenean leo ligula.\\
%\\
\\
\\
\textbf{Anmerkung 6 (Anwendungen, Allgemein):}
\newline
$\bullet$ Die Technik des Geoinformationssystems wurde durch zwei Arten(Dijkstra und verbesserter Suchbereich) erläutert. Während die Technik eines Roboters wurde nicht in einem Beispiel realisiert.
\newline
$\bullet$ Ich vermute, dass eine Unterscheidung zwischen traditioneller Dijkstra und verbesserter Dijkstra nicht in dem Anwendungsteil zu finden soll. 
\newline 
%\textbf{Stellungnahme zu 3:}
%Lorem ipsum dolor sit amet, consectetuer adipiscing elit. Aenean commodo ligula eget dolor. Vivamus elementum semper nisi. Aenean vulputate eleifend tellus. Aenean leo ligula.\\
%\\
 \\
 \\
\\
\textbf{Anmerkung 7 (Kleinigkeiten):}
Mit oder ohne Absicht weiß ich nicht aber mehrere Leerzeichen sind falsch gestellt.
\begin{enumerate}
	\item  (Seite 1, Zeile 7)
	\item  (Seite 1, Zeile 10)
	\item  (Seite 8, 3.2 Informed Search, Zeile 10)
	\item  (Seite 9, Zeile 3)
	\item  (Seite 9, Zeile 11)
	\item  (Seite 9, Zeile 14)
	\item  (Seite 9, Zeile 10)
	\item  (Seite 9, 3.3.1 ALT-Algorithmen, Zeile 4)
	\item  (Seite 10, Zeile 1)
\end{enumerate}
-Es gibt eigentlich noch viele.
\\ \\
\textbf{Antwort:}
Damit meinst du wahrscheinlich die Absätze. Die sind absichtlich so formatiert, dass das erste Wort nach einem Absatz eingerückt ist.
\\
%\textbf{Stellungnahme zu 3:}
%Lorem ipsum dolor sit amet, consectetuer adipiscing elit. Aenean commodo ligula eget dolor. Vivamus elementum semper nisi. Aenean vulputate eleifend tellus. Aenean leo ligula.\\
%\\
\end{document}