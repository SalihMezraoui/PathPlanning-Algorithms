%%%%%%%%%%%%%%%%%%% vorlage.tex %%%%%%%%%%%%%%%%%%%%%%%%%%%%%
%
% LaTeX-Vorlage zur Erstellung von Projekt-Dokumentationen
% im Fachbereich Informatik der Hochschule Trier
%
% Basis: Vorlage svmono des Springer Verlags
%
%%%%%%%%%%%%%%%%%%%%%%%%%%%%%%%%%%%%%%%%%%%%%%%%%%%%%%%%%%%%%

\documentclass[envcountsame,envcountchap, deutsch]{i-studis}

\usepackage{makeidx}         	% Index
\usepackage{multicol}        	% Zweispaltiger Index
%\usepackage[bottom]{footmisc}	% Erzeugung von Fußnoten

%%-----------------------------------------------------
%\newif\ifpdf
%\ifx\pdfoutput\undefined
%\pdffalse
%\else
%\pdfoutput=1
%\pdftrue
%\fi
%%--------------------------------------------------------
%\ifpdf
\usepackage[pdftex]{graphicx}
\usepackage{epstopdf}
\usepackage[pdftex,plainpages=false]{hyperref}
%\else
%\usepackage{graphicx}
%\usepackage[plainpages=false]{hyperref}
%\fi

%%-----------------------------------------------------
\usepackage{color}				% Farbverwaltung
%\usepackage{ngerman} 			% Neue deutsche Rechtsschreibung
\usepackage[english, ngerman]{babel}
%\usepackage[latin1]{inputenc} 	% Ermöglicht Umlaute-Darstellung
%\usepackage[utf8]{inputenc}  	% Ermöglicht Umlaute-Darstellung unter Linux (je nach verwendetem Format)
\usepackage[T1]{fontenc}
\usepackage{textcomp}

%-----------------------------------------------------
\usepackage{listings} 			% Code-Darstellung
\lstset
{
	basicstyle=\scriptsize, 	% print whole listing small
	keywordstyle=\color{blue}\bfseries,
								% underlined bold black keywords
	identifierstyle=, 			% nothing happens
	commentstyle=\color{red}, 	% white comments
	stringstyle=\ttfamily, 		% typewriter type for strings
	showstringspaces=false, 	% no special string spaces
	framexleftmargin=7mm, 
	tabsize=3,
	showtabs=false,
	frame=single, 
	rulesepcolor=\color{blue},
	numbers=left,
	linewidth=146mm,
	xleftmargin=8mm
}
\usepackage{textcomp} 			% Celsius-Darstellung
\usepackage{amssymb,amsfonts,amstext,amsmath}	% Mathematische Symbole
\usepackage[german, ruled, vlined]{algorithm2e}
\usepackage[a4paper]{geometry} % Andere Formatierung
\usepackage{bibgerm}
\usepackage{array}
\hyphenation{Ele-men-tar-ob-jek-te  ab-ge-tas-tet Aus-wer-tung House-holder-Matrix Le-ast-Squa-res-Al-go-ri-th-men} 		% Weitere Silbentrennung bei Bedarf angeben
\setlength{\textheight}{1.1\textheight}
\pagestyle{myheadings} 			% Erzeugt selbstdefinierte Kopfzeile
\makeindex 						% Index-Erstellung


%--------------------------------------------------------------------------
\begin{document}
%------------------------- Titelblatt -------------------------------------
\title{Funktionsprinzipien und Anwendungen von Algorithmen zur Pfadplanung}
\project{Ausarbeitung zur Vorlesung Wissenschaftliches Arbeiten}
%--------------------------------------------------------------------------
\supervisor{Titel Vorname Name} 		% Betreuer der Arbeit
\author{Bearbeiter 1: Mohammed Salih Mezraoui \\Bearbeiter 2: David Gruber \\Bearbeiter 3: Marius Müller}							% Autor der Arbeit
\groupid{WissArb22/Thema 2/Gruppe-7}
\address{Trier,} 							% Im Zusammenhang mit dem Datum wird hinter dem Ort ein Komma angegeben
\submitdate{15.07.2022} 				% Abgabedatum
%\begingroup
%  \renewcommand{\thepage}{title}
%  \mytitlepage
%  \newpage
%\endgroup
\begingroup
  \renewcommand{\thepage}{Titel}
  \mytitlepage
  \newpage
\endgroup
%--------------------------------------------------------------------------
\frontmatter 
%--------------------------------------------------------------------------
\kurzfassung

%% deutsch
\paragraph*{}
In dieser wissenschaftlichen Arbeit werden Algorithmen zur Pfadplanung dargestellt und analysiert. Es wird anhand der Anwendung in Geoinformationssystemen und bei mobilen Robotern aufgezeigt, wie der Algorithmus von Dijkstra eingesetzt wird und welche Stärken und Schwächen damit einhergehen.  Da die Fragestellungen, die mit Algorithmen zur Pfadplanung bearbeitet werden, immer komplexer werden, steigen auch die Anforderungen an Zeit- und Platzkomplexität. Daher werden in dieser Arbeit die wichtigsten Optimierungsstrategien für Algorithmen zur Pfadplanung dargestellt. In einer experimentellen Analyse konnte gezeigt werden, dass die Laufzeit des Algorithmus von Dijkstra durch den Einsatz von Preprocessing und heuristischer Funktionen um mehrere Größenordnungen gesenkt werden kann. 


 			% Kurzfassung Deutsch/English
\tableofcontents 						% Inhaltsverzeichnis
%--------------------------------------------------------------------------
\mainmatter                        		% Hauptteil (ab hier arab. Seitenzahlen)
%--------------------------------------------------------------------------
% Die Kapitel werden in separaten .tex-Dateien abgelegt und hier eingebunden.
\chapter{Einleitung und Problemstellung}

Begonnen werden soll mit einer Einleitung zum Thema, also Hintergrund und Ziel erläutert werden.

Weiterhin wird das vorliegende Problem diskutiert: Was ist zu lösen, warum ist es wichtig, dass man dieses Problem löst und welche Lösungsansätze gibt es bereits. Der Bezug auf vorhandene oder eben bisher fehlende Lösungen begründet auch die Intention und Bedeutung dieser Arbeit. Dies können allgemeine Gesichtspunkte sein: Man liefert einen Beitrag für ein generelles Problem oder man hat eine spezielle Systemumgebung oder ein spezielles Produkt (z.B. in einem Unternehmen), woraus sich dieses noch zu lösende Problem ergibt.

Im weiteren Verlauf wird die Problemstellung konkret dargestellt: Was ist spezifisch zu lösen? Welche Randbedingungen sind gegeben und was ist die Zielsetzung? Letztere soll das
beschreiben, was man mit dieser Arbeit (mindestens) erreichen möchte.
\chapter{Algorithmen zur Pfadplanung}\index{Algorithmen}\index{Pfadplanung}\index{Algorithmus}
In diesem Kapitel werden die Algorithmen \emph{Dijkstra} und \emph{Bellman-Ford} zur Pfadplanung beschrieben.

\section{Einleitung}
\label{Was ist Pfadplanung?}
Die Pfadplanung ist ein nichtdeterministisches Problem mit polynomialer Laufzeit ("NP"), bei dem es darum geht, einen Pfad zu finden, 
der in einem System den Ausgangspunkt mit dem Ziel verbindet. Der zu wählende Weg (die beste Route) wird durch Beschränkungen 
und Bedingungen bestimmt \cite{Karur:21}.

In der Informatik, insbesondere in der Graphentheorie, ist das Problem des kürzesten Weges bekannt. Der kürzestmögliche Weg von 
einer Quelle zu einem Ziel hat die geringsten Längenanforderung.

Die Dijkstra- und Bellman-Algorithmen für den kürzesten Weg sind in einer Vielzahl von Bereichen und Anwendungen
weit verbreitet. Beispiele für diese Anwendungen sind Routing-Protokolle für Netzwerke, Routenplanung, 
Verkehrssteuerung, Pfadfindung in sozialen Netzwerken, Computerspiele und Navigationssysteme \cite{Panda:18}.

\section{Uninformierter Ansatz}
\label{Uninformierter Ansatz}
Als Einstieg in die Thematik der Pfadplanung wird der Uninformierte Ansatz beschrieben.
Uninformierte Algorithmen werden auch „blind“ genannt, weil bei ihrer 
Suche auf keine zusätzlichen Informationen (wie z.B. Gewichtungen) zurückgegriffen wird \cite[81]{Russell:10}.
\subsection{Breitensuche}
\label{Breitensuche}
Die Breitensuche gehört zu den uninformierten Suchalgorithmen. 
Der erste Schritt der Breitensuche ist es alle verbundenen Knoten ersten Grades des Wurzelknotens zu besuchen.
Dies wird Ebene für Ebene im Baum wiederholt, bis alle Knoten besucht wurden. 
Die Breitensuche findet weitestgehend in der Graphentheorie ihre Anwendung \cite[81]{Russell:10}.
\\
\\
\\
\\
\subsection{Tiefensuche}
\label{Tiefensuche}
Die Tiefensuche gehört ebenfalls zu den uninformierten Suchalgorithmen.
Das Vorgehen bei der Tiefensuche ist es zuerst alle Unterknoten eines direkten Nachbarn des Wurzelknotens zu besuchen bis dieser 
keine Unterknoten mehr hat.
Erst dann wird der nächste Nachbar in der ersten Ebene besucht bis keine unbesuchten Knoten mehr vorhanden sind \cite[85,86]{Russell:10}. 
Die Tiefensuche ist indirekt an vielen komplexeren Algorithmen beteiligt. 
So nutzt die iterative Vertiefungssuche die klassische Tiefensuche in Wiederholung in Verbindung mit einer inkrementellen 
Tiefenbegrenzung bis ein Ziel gefunden wird \cite[108,109]{Russell:10}.
Außerdem kann die Tiefensuche auch für das Ermitteln von Zusammenhangskomponenten
oder für das Erzeugen eines Irrgartens verwendet werden \cite[85,86]{Russell:10}.

\section{Dijkstra Algorithmus}
\label{Dijkstra Algorithmus}

Der Dijkstra-Algorithmus (benannt nach seinem Entdecker E.W. Dijkstra) ist ein bekannter Algorithmus auf dem Gebiet der optimalen
Pfadwahl und wird verwendet, um den kürzesten Pfad von einem Startpunkt in einem Graphen zu einem Zielpunkt zu finden \cite{Javaid2019}.

\subsection{Funktionsprinzip}

Das Grundkonzept des Algorithmus ist wie folgt:
\\ \\
In einem ersten Schritt wird der Startknoten festgelegt und der Abstand zwischen ihm und den anderen Knoten des Graphen berechnet. 
Gibt es keine Kante, die diesen Knoten mit dem Startpunkt verbindet, ist der Abstand unendlich; gibt es eine Kante, die 
diesen Knoten mit dem Startpunkt verbindet, ist der Abstand $n$; und das niedrigste Gewicht (wenn es mehrere Kanten gibt) ist $n$.

Der zweite Schritt besteht darin, den Punkt mit dem kürzesten Abstand zum Startpunkt zu ermitteln und zu speichern. Die zuvor ermittelte 
Entfernung wird mit derjenigen verglichen, die über den soeben beiseite gelegten Punkt für alle verbleibenden Punkte 
ermittelt wurde, und nur der kleinste der beiden Werte wird beibehalten. Dieser Vorgang wird so lange wiederholt, bis es keine Punkte
mehr gibt oder bis der Ankunftspunkt gewählt ist \cite{Zhou:19}.
\\ \\
Im Folgenden wird die Umsetzung des unten stehenden Pseudocodes  \ref{Dijkstra Algorithmus Pseudocode} erläutert, angelehnt an \cite{Abusalim2020}.

Beim Dijkstra-Algorithmus ist die Route unbekannt. Die Knoten werden als temporär $(t)$ oder permanent $(p)$ eingestuft.
Ein temporäres Label wird geändert, wenn eine kürzere Route zu einem Knoten gefunden werden kann. Wenn keine bessere Route gefunden werden kann, wird der Status des temporären Labels in permanent geändert.
\begin{itemize}
	\item Schritt1: die Entfernung des Quellknotens wird der Wert Null zugewiesen $[distance (source) = 0]$, und die Entfernung der anderen
		Knoten wird auf unendlich gesetzt $[distance(x) = Infinity]$.
	\item Schritt 2: Suche des Knotens $x$ mit der kürzesten Entfernung $d(x)$. Wenn $d(x)$ unendlich ist oder es keine temporären Knoten gibt,
		wurde der Knoten $x$ als permanent markiert, was bedeutet, dass sich $d(x)$ und sein übergeordneter Wert nicht ändern werden.
	\item Schritt 3: Für jeden temporären Knoten mit dem Label $y$, der an $x$ angrenzt, wird der folgende Vergleich durchgeführt:

	
\end{itemize}
% \begin{gather*} \label{eq:2.1}
wenn
\begin{equation} \label{eq:2.1}
	d(x) + w (x, y) < d(y)	
\end{equation}
dann
\begin{equation} \label{eq:2.2}
	D(y) = d(x) + w (x, y)
\end{equation}
\newline
Wenn die Entfernung des gekennzeichneten Knotens $d(y)$ größer ist als die Entfernung des gekennzeichneten Knotens $d(x)$ plus Verbindungsgewicht $w(x, y)$, 
wird die Entfernung des gekennzeichneten Knotens $y$ gemäß den Gleichungen \ref{eq:2.1} und \ref{eq:2.2} aktualisiert in $D(y)$.

\noindent \\
\begin{minipage}{1.0\textwidth} \small
\begin{lstlisting}
$\textbf{function}$ DIJKSTRA (Graph, source)
	create vertex set D
	for each vertex v in Graph:
		distance[v] $\leftarrow$ INFINITY
		previous[v] $\leftarrow$ UNDEFINED
		add v to D
	distance[source] $\leftarrow$ 0
	
	while D is not empty do:
	    u $\leftarrow$ in D with min distance[u]
	    remove u from D
	    for each neighbour v of u:
	          alt  $\leftarrow$ distance[u] + length(u,v)
	          if alt < distance[v]
	              distance[v]  $\leftarrow$  alt
	              previous[v]  $\leftarrow$  u
	end while
	return distance[], previous[]
end function

\end{lstlisting}
\captionof{lstlisting}{Dijkstra-Algorithmus Pseudocode, angelehnt an \cite{Abusalim2020}.}
 \label{Dijkstra Algorithmus Pseudocode}
\end{minipage}


\subsection{Vor- und Nachteile}
Der Dijkstra-Algorithmus hat zwei wesentliche Vorteile: Er kann alle optimalen Pfade finden und die Trefferquote dafür liegt
bei 100\%. Der zweite Vorteil ist, dass er die verbleibenden unerwünschten Knoten nicht besucht, wenn der beabsichtigte Zielknoten erreicht
ist \cite{Zhou:19,Abusalim2020}.
\\ \\
Der Hauptnachteil des Algorithmus besteht darin, dass er eine blinde Suche durchführt, was eine erhebliche Menge an Zeit und Ressourcen vergeudet. 
Ein weiterer Nachteil ist, dass er nicht mit negativen Kanten umgehen kann, was zu azyklischen Graphen führt, und daher nicht immer den kürzesten 
Weg findet \cite{Mukhlif2020}.


\section{Bellman-Ford-Algorithmus}
\label{Bellman-Ford-Algorithmus}

In 1958 veröffentlichte Richard Bellman den Bellman-Ford-Algorithmus, einen Graphen-Suchalgorithmus zur Ermittlung des kürzesten 
Pfades \cite{Abusalim2020,Sulaiman18}.
\subsection{Eigenschaften}
Um kürzeste Wege auf gerichteten Graphen zu finden, nutzt der Bellman-Ford-Algorithmus die Relaxation \cite{Vaibhavi2014}.
Relaxation bedeutet, dass geprüft wird, ob es möglich ist, den Weg zu dem Knoten, auf den die Kante zeigt, zu verkürzen, 
und wenn dies der Fall ist, wird der Weg zum Knoten durch den entdeckten Weg ersetzt \cite{Abusalim2020}.
Der Bellman-Ford Algorithmus kann auch mit negative Kantengewichten umgehen. 
Wenn es Zyklen mit negativem Gewicht gibt, wird der Algorithmus sie erkennen (so dass es keine Lösung gibt) \cite{Vaibhavi2014}.

Die Vorteile dieses Algorithmus sind unter anderem, dass es sich um einen dynamischen Algorithmus handelt, dass er negative gerichtete 
Kanten (und auch positive) berechnen kann, dass er die Kosten für den Aufbau des Netzes reduzieren kann, indem er den kürzesten Weg von einem 
Knoten zum anderen findet, und dass er die Anzahl der aufgebauten Router-Pfad reduzieren kann \cite{Abusalim2020}.

\subsection{Funktionsprinzip}\index{Bellman-Ford}

Angelehnt an \cite{Abusalim2020} wird der Bellman-Ford-Algorithmus wie folgt ausgeführt:
\\
\begin{itemize}
	\item Schritt 1: Zuweisung des Abstandswertes des Startpunktes $s$ zu null $(distance[s] = 0)$ und des Abstandswertes der anderen Punkte zu 
		$INFINITY$.
	\item Schritt 2: Wenn $n$ die Anzahl der Knoten ist, wird jede Kante $(n - 1)$ Mal relaxiert. 
	\item Schritt 3: Mit der N-ten Schleife wird geprüft, ob der Graph negative Zyklen aufweist.
\end{itemize}
Der Bellman-Ford-Algorithmus wird wie in Listing  \ref{lst:Bellman-Ford Pseudo-code} unten dargestellt ausgeführt.

\noindent \\
\begin{minipage}{1.0\textwidth} \small
\begin{lstlisting}[label=lst:Bellman-Ford]
$\textbf{function}$ bellmanFord (G, s)
	for each vertex V in G:
		distance[v] $\leftarrow$ INFINITY
		previous[v] $\leftarrow$ NULL
	distance[s] $\leftarrow$ 0
	for each vertex V in G
	    for each edge(u,v) in G
	         alt  $\leftarrow$ distance[u] + length(u,v)
	         if alt < distance[v]
	             distance[v] $\leftarrow$ alt
	             previous[v] $\leftarrow$ u
	 
	for each edge(u,v) in G
	       if distance[u] + length(u,v) < distance[v]
	               Error: Negative Cycle exists
return distance[], previous[]
end function
       
\end{lstlisting}
\captionof{lstlisting}{Bellman-Ford-Algorithmus Pseudocode, angelehnt an \cite{Abusalim2020}.}
\label{lst:Bellman-Ford Pseudo-code}
\end{minipage}








\chapter{Optimierungsstrategien}
\chapter{Anwendungen}
\label{Anwendungen}

In diesem Kapitel wird beschrieben, wie Pfadplanung Algorithmen in realen Anwendungen eingesetzt werden können.
\section{GIS(Geoinformationssystem)}
\label{GIS(Geoinformationssystem)}
% \subsection{Definition}

Ein geografisches Informationssystem (GIS) ist ein System zum Erstellen, Verwalten, Analysieren und Kartieren verschiedener Arten von Daten. GIS verknüpft Daten mit einer Karte und integriert Standortdaten mit verschiedenen Arten von beschreibenden Daten. Dies ist die Grundlage für die Kartierung und Analyse in der Wissenschaft und in fast allen Branchen.
Bei diesen Daten handelt es sich um Informationen über Objekte auf der Erde wie Städte, Eisenbahnstrecken, Flüsse usw \cite{Vaibhavi2014}. 

% \subsection{Geoinformationssystem mit Dijkstra-Algorithmus}
Algorithmen für den kürzesten Weg werden in Kartenplattformen wie Google Maps verwendet, um den kürzesten Weg zwischen zwei Punkten zu ermitteln.
Obwohl der Dijkstra-Standardalgorithmus in diesem Fall anwendbar zu sein scheint, dauert das Routing des Pfades vom Startpunkt zum Endpunkt in einer großen Datenmenge sehr lange \cite{HamidAli2020}.

 Um bei der Berechnung des kürzesten Weges Zeit zu sparen und eine bessere Lösung zu erhalten. Der bessere Ansatz besteht darin, angelehnt an \cite{HamidAli2020}. Vor dem Start des Dijkstra-Algorithmus einen temporären Datensatz zu erstellen und ihn so zu behandeln, als wäre er der Graph, mit dem der Algorithmus arbeiten muss.
 Nachdem die Start- und Endknoten ermittelt wurden, wird dieser Datensatz erstellt. Der Datensatz kann anhand der Koordinaten des Start- und des Zielknotens aus den Hauptdaten ausgeschlossen werden, indem nur die Knoten ausgewählt werden, die sich innerhalb des von den beiden Knoten gebildeten reduzierten Suchbereichs befinden.
 


\newpage
\section{Mobile Roboter}
\label{Mobile Roboter}
Die Steuerung mobiler Roboter ist ein wichtiges Thema in intelligenten Städten und die Pfadplanung ist eine wichtige Komponente unter den mobilen Roboter-Technologien, die dem Roboter hilft, den Weg von einem Startpunkt zu einem Zielpunkt zu finden und dabei sicher und zuverlässig alle Hindernisse in einer statischen und dynamischen Umgebung während der Reise zu vermeiden \cite{Myung21,Hong-mei17}.

\subsection{Umgebungsmodell}
Das Rolling-Window-Prinzip wählt einen lokal optimalen Zielzustand innerhalb des Erfassungsbereichs der Sensoren, wenn eine mögliche Kollision vorhergesagt wird, während sich der Roboter auf dem Weg in einer unbekannten Umgebung bewegt, während die Algorithmen Dijkstra und A* in erster Linie für die Pfadplanung in einer bekannten Umgebung eingesetzt werden.
\newline
Da sowohl der Dijkstra- als auch der A*-Algorithmus Gitter-basierte Suchmethoden sind, sollte zunächst das Gittermodell der umgebenden Karte erstellt werden. Bei der Gittermethode wird die Karte in benachbarte Gitter gleicher Größe unterteilt. 
Die Größe des mobilen Roboters bestimmt die Größe des Gitters und beeinflusst die Suchgenauigkeit und Effizienz des Algorithmus \cite{Hong-mei17}.

\subsection{Dijkstras Algorithmus und A*-Algorithmus}

Die Idee ist, A*- und Dijkstra-Algorithmen miteinander zu kombinieren, um die Effizienz der Planung einer idealen kollisionsfreien Route für einen mobilen Roboter zu erhöhen. Bei der Neuplanung wird die Dijkstra-Methode zur Vorverarbeitung der statischen Umgebungskartendaten angewendet, die optimale Routen vom Ziel zu allen freien Zuständen plant und speichert. Wenn der Roboter auf seinem Weg zum Ziel auf ein bewegliches Hindernis trifft, wählt das Rolling-Window-Prinzip einen lokal optimalen Zustand als nächsten Zielzustand. Dann wird mit dem A*-Algorithmus eine lokal optimale Route von der Roboterposition zum lokalen Ziel neu geplant.
Die neue Route kann den Roboter um Hindernisse herumführen \cite{Hong-mei17}.

\newpage

\section{Autonome Navigation}
\label{Autonome Navigation}

Pfadsuchalgorithmen werden ebenfalls im Bereich des Autonomen Fahrens, oder auch bei unbemannten Flugfahrzeugen verwendet,
um sichere, effiziente, kollisionsfreie und kostengünstige Wege von Start zum Ziel zu führen, was die Wahl des richtigen Pfadsuchalgorithmus
zu einer wichtigen Aufgabe macht. Es hängt unter Anderem die Geometrie des Fahrzeugs von dieser Wahl ab \cite{Karur:21}.
Mit der zunehmenden Verbreitung von autonomen Fahrzeugen, die immer mehr Wegfindung und -planung erfordert, sind Pfadsuchalgorithmen
zu einem neuen Schwerpunkt der autonomen Steuerung geworden \cite{Karur:21}.
Da mobile Roboter in vielen Anwendungen eingesetzt werden, haben Forscher Methoden entwickelt, um die 
Anforderungen an mobile Roboter effektiv erfüllen zu können und einige Herausforderungen für die Umsetzung einer vollständig oder
teilweise autonomen Navigation in unübersichtlichen Umgebungen zu bewältigen \cite{Karur:21}.
So wird die Wahl des richtigen Pfadplanungsalgorithmus von der kinematischen Bewegungsgestaltung des Roboters/Fahrzeugs,
den zur Verfügung stehenden Rechenressourcen, sowie der sensorischen Ausstattung des Fahrzeugs bestimmt \cite{Karur:21}.

Die Leistung und Komplexität des verwendeten Algorithmus hängt vom Anwendungsfall ab \cite{Karur:21}.

Es gibt nicht \emph{den perfekten Pfadsuchalgorithmus für Autonomes Fahren}, es finden aber viele verschiedene Algorithmen eine 
Anwendung in der autonomen Navigation.
\chapter{Zusammenfassung und Ausblick}
\label{Zusammenfassung und Ausblick}
In den letzten Jahren hat die Pfadplanung immer mehr an Relevanz gewonnen, zum Beispiel wird beim autonomen Autofahren immer mehr Forschung 
im Bereich der Pfadplanungsalgorithmen betrieben \cite{Karur:21}.
\noindent \\
\begin{itemize}
    \item Bevor ein Programm mit der Suche nach der besten Lösung (dem besten Weg) beginnen kann, muss erst ein Ziel deklariert und das Problem (Umgebung) genau definiert werden \cite[108,109]{Russell:10}.
    \item Suchalgorithmen betrachten Zustände und Aktionen atomar, das heißt sie berücksichtigen keine interne Struktur, die sie besitzen könnten \cite[108,109]{Russell:10}.
    \item Sie werden nach folgenden Kriterien bewertet: Optimalität, Vollständigkeit, sowie Raum- und Zeitkomplexität \cite[80]{Russell:10}.
    \item Uninformierte Pfadsuchalgorithmen verfügen nur über eine grundlegende Problemdefinition und keine weiteren Metriken/Heuristiken. In dieser wissenschaftlichen Arbeit wurden folgende uninformierte Algorithmen vorgestellt:
    \begin{itemize}
        \item Die Breitensuche (Kapitel \ref{Breitensuche}) expandiert zuerst die flachsten Knoten. Sie ist vollständig, optimal für einheitliche Pfadkosten, hat jedoch eine exponentielle Raumkomplexität \cite[81]{Russell:10}.
        \item Die Tiefensuche (Kapitel \ref{Tiefensuche}) expandiert zuerst den tiefsten nicht expandierten Knoten. Sie ist weder vollständig noch optimal, hat aber eine lineare Raumkomplexität \cite[85,86]{Russell:10}.
        \item Die iterative Vertiefungssuche (Kapitel \ref{Tiefensuche}) ist eine Wiederholung der Tiefensuche mit zunehmender Tiefenbegrenzung, bis ein Ziel gefunden wird. Sie ist vollständig, optimal für die Kosten pro Schritt, hat eine vergleichbare Zeitkomplexität wie die Breitensuche und eine lineare Raumkomplexität \cite[85,86]{Russell:10}.
        \item Der Greedy Dijkstra-Algorithmus (Kapitel \ref{Dijkstra-Algorithmus}) der zwar bei der blinden Suche Zeit vergeudet, aber dafür optimal ist und eine Trefferquote von 100\% hat \cite{Karur:21}.
        \item Die Optimierung durch eine bidirektionale Suche kann die Zeitkomplexität reduzieren, allerdings ist sie nicht immer für das Problem geeignet und kann viel Speicherplatz beanspruchen \cite[108,109]{Russell:10}.\\\\
    \end{itemize}
    \newpage
    \item Informierte Suchmethoden basieren auf heuristischen Funktionen, die die Kosten einer Lösung schätzen können \cite[108,109]{Russell:10}. 
    \begin{itemize}
        \item Der Greedy Best-First-Search-Algorithmus (Kapitel \ref{Optimierungsstrategien}) expandiert Knoten nach einem minimalen heuristischen Funktionswert. Er ist nicht optimal, aber dafür effizient \cite[108,109]{Russell:10}.
        \item A*-Suche (Kapitel \ref{A*}) expandiert Knoten mit minimalem Heuristischen Funktions- und Pfadkostenwerten. A* ist vollständig und optimal, vorausgesetzt die heuristische Funktion ist zulässig.
        \item ALT-Algorithmen (Kapitel \ref{ALT-Algorithmen}), die aufgebaut auf A* durch Preprocessing noch optimiertere Ergebnisse erzeugt.
        \item Reach-based Pruning (Kapitel \ref{Reach-Based Pruning}), welches den Dijkstra-Algorithmus um eine Metrik erweitert und dadurch optimiert.
    \end{itemize}
    \item Wie ein, durch eine heuristische Funktion optimierter, informierter Pfadsuchalgorithmus leistungsbezogen abschneidet, hängt von der Qualität der heuristischen Funktion ab \cite[108,109]{Russell:10}.
\end{itemize}

% ...
%--------------------------------------------------------------------------
\backmatter                        		% Anhang
%-------------------------------------------------------------------------
\bibliographystyle{geralpha}			% Literaturverzeichnis
\bibliography{literatur}     			% BibTeX-File literatur.bib
%--------------------------------------------------------------------------
\printindex 							% Index (optional)
%--------------------------------------------------------------------------
\begin{appendix}						% Anhänge sind i.d.R. optional
   \chapter{Glossar}

\abbreviation{ALT}{A* search, Landmarks und Triangle inequality}
\abbreviation{NP}{nichtdeterministisches, polynomialzeitliches}
% \abbreviation{DisASTer}		{DisASTer (Distributed Algorithms Simulation Terrain), A platform for the Implementation of Distributed Algorithms}
% \abbreviation{DSM}			{Distributed Shared Memory}
% \abbreviation{AC}			{Linearisierbarkeit (atomic consistency)}
% \abbreviation{SC}			{Sequentielle Konsistenz (sequential consistency)}
% \abbreviation{WC}			{Schwache Konsistenz (weak consistency)}
% \abbreviation{RC}			{Freigabekonsistenz (release consistency)}
			% Glossar   
   \chapter*{Arbeitsverteilung}

\textbf{Teilnehmer 1: Mohammed Salih Mezraoui} 
\\
 Inhalte:
 \begin{itemize}
    \item Kapitel \ref{Was ist Pfadplanung?}: Was ist Pfadplanung?
    \item Kapitel \ref{Dijkstra Algorithmus}: Dijkstra Algorithmus
    \item Kapitel \ref{Bellman-Ford-Algorithmus}: Bellman-Ford-Algorithmus
    \item Kapitel \ref{GIS(Geoinformationssystem)}: GIS(Geoinformationssystem)
    \item Kapitel \ref{Mobile Roboter}: Mobile Roboter
\end{itemize}
\paragraph*{}
\textbf{Teilnehmer 2: David Gruber} 
\\
Inhalte: 
\begin{itemize}
    \item Kapitel \ref{Einleitung und Problemstellung}: Einleitung und Problemstellung
    \item Kapitel \ref{Uninformierter Ansatz}: Uninformierter Ansatz
    \item Kapitel \ref{Autonome Navigation}: Autonome Navigation
    \item Kapitel \ref{Zusammenfassung und Ausblick}: Zusammenfassung und Ausblick
\end{itemize}
\paragraph*{}
\textbf{Teilnehmer 3: Marius Müller} 
\\
 Inhalte: 
\begin{itemize}
	\item Kapitel \ref{Optimierungsstrategien}: Optimierungsstrategien
\end{itemize}		% Welcher Studierende hat welche Texte formuliert?
   
\end{appendix}

\end{document}
