\kurzfassung

%% deutsch
\paragraph*{}
In dieser wissenschaftlichen Arbeit werden Algorithmen zur Pfadplanung dargestellt und analysiert. Es wird anhand der Anwendung in Geoinformationssystemen und bei mobilen Robotern aufgezeigt, wie der Algorithmus von Dijkstra eingesetzt wird und welche Stärken und Schwächen damit einhergehen.  Da die Fragestellungen, die mit Algorithmen zur Pfadplanung bearbeitet werden, immer komplexer werden, steigen auch die Anforderungen an Zeit- und Platzkomplexität. Daher werden in dieser Arbeit die wichtigsten Optimierungsstrategien für Algorithmen zur Pfadplanung dargestellt. In einer experimentellen Analyse konnte gezeigt werden, dass die Laufzeit des Algorithmus von Dijkstra durch den Einsatz von Preprocessing und heuristischer Funktionen um mehrere Größenordnungen gesenkt werden kann. 


