%%%%%%%%%%%%%%%%%%% vorlage.tex %%%%%%%%%%%%%%%%%%%%%%%%%%%%%
%
% LaTeX-Vorlage zur Erstellung von Projekt-Dokumentationen
% im Fachbereich Informatik der Hochschule Trier
%
% Basis: Vorlage svmono des Springer Verlags
%
%%%%%%%%%%%%%%%%%%%%%%%%%%%%%%%%%%%%%%%%%%%%%%%%%%%%%%%%%%%%%

\documentclass[envcountsame,envcountchap, deutsch]{i-studis}

\usepackage{makeidx}         	% Index
\usepackage{multicol}        	% Zweispaltiger Index
%\usepackage[bottom]{footmisc}	% Erzeugung von Fußnoten

%%-----------------------------------------------------
%\newif\ifpdf
%\ifx\pdfoutput\undefined
%\pdffalse
%\else
%\pdfoutput=1
%\pdftrue
%\fi
%%--------------------------------------------------------
%\ifpdf
\usepackage[pdftex]{graphicx}
\usepackage{epstopdf}
\usepackage[pdftex,plainpages=false]{hyperref}
%\else
%\usepackage{graphicx}
%\usepackage[plainpages=false]{hyperref}
%\fi

%%-----------------------------------------------------
\usepackage{color}				% Farbverwaltung
%\usepackage{ngerman} 			% Neue deutsche Rechtsschreibung
\usepackage[english, ngerman]{babel}
%\usepackage[latin1]{inputenc} 	% Ermöglicht Umlaute-Darstellung
%\usepackage[utf8]{inputenc}  	% Ermöglicht Umlaute-Darstellung unter Linux (je nach verwendetem Format)
\usepackage[T1]{fontenc}
\usepackage{textcomp}

%-----------------------------------------------------
\usepackage{listings} 			% Code-Darstellung
\lstset
{
	basicstyle=\scriptsize, 	% print whole listing small
	keywordstyle=\color{blue}\bfseries,
								% underlined bold black keywords
	identifierstyle=, 			% nothing happens
	commentstyle=\color{red}, 	% white comments
	stringstyle=\ttfamily, 		% typewriter type for strings
	showstringspaces=false, 	% no special string spaces
	framexleftmargin=7mm, 
	tabsize=3,
	showtabs=false,
	frame=single, 
	rulesepcolor=\color{blue},
	numbers=left,
	linewidth=146mm,
	xleftmargin=8mm
}
\usepackage{textcomp} 			% Celsius-Darstellung
\usepackage{amssymb,amsfonts,amstext,amsmath}	% Mathematische Symbole
\usepackage[german, ruled, vlined]{algorithm2e}
\usepackage[a4paper]{geometry} % Andere Formatierung
\usepackage{bibgerm}
\usepackage{array}
\hyphenation{Ele-men-tar-ob-jek-te  ab-ge-tas-tet Aus-wer-tung House-holder-Matrix Le-ast-Squa-res-Al-go-ri-th-men} 		% Weitere Silbentrennung bei Bedarf angeben
\setlength{\textheight}{1.1\textheight}
\pagestyle{myheadings} 			% Erzeugt selbstdefinierte Kopfzeile
\makeindex 						% Index-Erstellung
\usepackage{float}

% Margin zwischen Tabelle und Tabellenunterschrift
\usepackage{caption} 
\captionsetup[table]{skip=10pt, labelfont+=bf}

%--------------------------------------------------------------------------
\begin{document}
%------------------------- Titelblatt -------------------------------------
\title{Funktionsprinzipien und Anwendungen von Algorithmen zur Pfadplanung}
\project{Ausarbeitung zur Vorlesung Wissenschaftliches Arbeiten}
%--------------------------------------------------------------------------
\supervisor{Titel Vorname Name} 		% Betreuer der Arbeit
\author{Bearbeiter 1: Mohammed Salih Mezraoui \\Bearbeiter 2: David Gruber \\Bearbeiter 3: Marius Müller}							% Autor der Arbeit
\groupid{WissArb22/Thema 2/Gruppe-7}
\address{Trier,} 							% Im Zusammenhang mit dem Datum wird hinter dem Ort ein Komma angegeben
\submitdate{15.07.2022} 				% Abgabedatum
%\begingroup
%  \renewcommand{\thepage}{title}
%  \mytitlepage
%  \newpage
%\endgroup
\begingroup
  \renewcommand{\thepage}{Titel}
  \mytitlepage
  \newpage
\endgroup
%--------------------------------------------------------------------------
\frontmatter 
%--------------------------------------------------------------------------
\kurzfassung

%% deutsch
\paragraph*{}
In dieser wissenschaftlichen Arbeit werden die wichtigsten Algorithmen zur Pfadplanung dargestellt und analysiert. Es wird anhand der Anwendung in Geoinformationssystemen und bei mobilen Robotern aufgezeigt, wie diese Algorithmen eingesetzt werden und welche Stärken und Schwächen damit einhergehen.  Da die Fragestellungen, die mit Algorithmen zur Pfadplanung bearbeitet werden, immer komplexer werden, steigen auch die Anforderungen an Zeit- und Platzkomplexität. Daher werden in dieser Arbeit die wichtigsten Optimierungsstrategien für Algorithmen zur Pfadplanung dargestellt. In einer experimentellen Analyse konnte gezeigt werden, dass die Laufzeit des Algorithmus von Dijkstra durch den Einsatz von Preprocessing und heuristischer Funktionen um mehrere Größenordnungen gesenkt werden kann. 

%% englisch
\paragraph*{}
In this scientific work, the most important algorithms for path planning are presented and analysed. The application in geoinformation systems and mobile robots shows how these algorithms are used and what strengths and weaknesses are associated with them. As the issues addressed by path planning algorithms are becoming increasingly complex, the demands on time and space complexity are also increasing. Therefore, this paper presents the most important optimization strategies for path planning algorithms. An experimental analysis showed that the runtime of the Dijkstra algorithm can be reduced by several orders of magnitude by using preprocessing and heuristic functions.
 			% Kurzfassung Deutsch/English
\tableofcontents 						% Inhaltsverzeichnis
%--------------------------------------------------------------------------
\mainmatter                        		% Hauptteil (ab hier arab. Seitenzahlen)
%--------------------------------------------------------------------------
% Die Kapitel werden in separaten .tex-Dateien abgelegt und hier eingebunden.

\chapter{Einleitung und Problemstellung}
\label{Einleitung und Problemstellung}


In diesem Paper werden wir aufeinander aufbauend Algorithmen vorstellen, die in einem Programm mit bestimmten Eingaben und Umgebungen 
wie z.B Graphen verwendet werden können um von einem gegebenen Start ein oder mehrere Ziele zu finden und dabei durch verschiedene Bewertungskriterien den besten Weg zu bestimmen \cite{Esri:00}.
Mit diesen Algorithmen kann ein Programm durch eine Sequenz von Aktionen sein Ziel erreichen. Diesen Prozess nennt man Suche \cite{Russell:10c}. \\


% Unter Pfadplanung versteht man die Berechnung des optimalen Weges zwischen einem Start- und einem oder mehreren Endpunkten in einem Netzwerk oder Graphen \cite{Esri:00}. \\
\noindent 
In diesem Paper werden wir verschiedene Pfadplanungsalgorithmen wie den Dijkstra-Algorithmus und der A* Algorithmus, 
die im Verkehrleitsystem von Google Maps verwendet werden, vorstellen \cite{Mehta:19}. 

% \noindent \\
% Um die Geschwindigkeit des Dijkstra-Algorithmus bei Blindsuchen zu optimieren werden A* und seine Varianten als Stand der Technik-Algorithmen 
% für den Einsatz in statischen Umgebungen vorgestellt \cite{Karur:21}.
\noindent \\
Das Thema \emph{Funktionsprinzipien und Anwendungen von Algorithmen zur Pfadplanung} hat, damals wie heute einen wichtigen Stellenwert. 
Ob im Bereich der Netzwerkroutenplanung oder bei KI-Spielern in Computerspielen: Pfadsuchalgorithmen sind so relevant wie nie \cite{Foeada:21}. 
Weitere Beispiele für die Anwendung von Pfadsuchalgroithmen sind Planung von öffentlichen Verkehrsmitteln und 
Robotik. Es werden zum Beispiel Pakete in Logistikzentren durch Roboter organisiert und Routen über verschiedene Logistikzentren durch Pfadsuchalgorithmen bestimmt.
Viele Bereiche im Alltag verwenden im Hintergrund Pfadsuchalgorithmen um den (kosten)günstigsten Weg zu finden. Dabei ist es wichtig, 
dass diese effizient, akkurat und schnell sein müssen, damit die Hauptsysteme noch genügen Ressourcen übrig haben um gewünscht zu funktionieren \cite{Foeada:21}. 
\noindent \\
Wir werden uns in diesem Paper die Funktionsweise der wichtigsten Pfadsuchalgorithmen, angefangen mit den uninformierten Suchalgorithmen wie der Breiten- oder Tiefensuche, 
welche einen ersten Einblick in die Thematik geben sollen und die Rahmenbedingungen und Problemumgebungen veranschaulichen sollen; 
Über den Bellman-Ford-Algorithmus, der nahezu intuitiv den kürzesten Weg liefert und bei dem auch negative Kantenlängen möglich sind\cite{Mukhlif:20}. 
Bis hin zu durch Heuristiken optimierte und informierte Pfadsuchalgorithmen wie Dijkstra oder A*, und wie sie in heutiger Zeit 
sonst eingesetzt werden können, veranschaulichen und jeweils (Pseudo-)Code- und Anwendungsbeispiele geben\cite{Russell:10a}.

\chapter{Algorithmen zur Pfadplanung}\index{Algorithmen}\index{Pfadplanung}\index{Algorithmus}
In diesem Kapitel werden zwei der am häufigsten verwendeten Algorithmen (Dijkstra und Bellman) für die Planung von Pfaden beschrieben.

\section{Was ist Pfadplanung?}
\label{Was ist Pfadplanung?}
Karthik Karur, Nitin Sharma, Chinmay Dharmatti, und Joshua E. Siegel haben Pfadplanung definiert als „ein nicht-deterministisches Polynomialzeit ("NP") schweres Problem definiert, dessen Aufgabe es ist, einen kontinuierlichen Pfad zu finden, der ein System von einer Anfangs- zu einer Endkonfiguration verbindet“\footnote{Definition der Pfadplanung \cite{Karur2021}}.
\newline
\newline
Die Komplexität des Problems wächst mit der Zunahme der Freiheitsgrade des Systems. Die zu wählende Route (die beste Route) wird durch Einschränkungen und Begrenzungen bestimmt, wie z. B. die kürzeste Entfernung zwischen den Endpunkten oder die kürzeste Zeit für eine kollisionsfreie Fahrt. Gelegentlich werden Beschränkungen und Ziele kombiniert, z. B. der Versuch, den Energieverbrauch zu begrenzen und gleichzeitig einen bestimmten Schwellenwert für die Fahrzeit nicht zu überschreiten.\cite{Karur2021}.

\section{Uninformierter Ansatz}\index{uninformiert}
\label{Uninformierter Ansatz}
\subsection{Breitensuche}\index{Breitensuche}
Die Breitensuche gehört zu den uninformierten Suchalgorithmen, diese werden auch „blind“ genannt, weil bei ihrer 
Suche auf keine zusätzlichen Informationen (wie z.B. Wichtungen) zurückgegriffen wird.
Bei der Breitensuche wird zunächst vom Wurzelknoten aus betrachtet alle verbunden Knoten ersten Grades besucht
und dies Ebene für Ebene im Baum\index{Baum} wiederholt bis alle Knoten\index{Knoten} besucht wurden. 
Die Breitensuche findet weitestgehend in der Graphentheorie seine Anwendung.\cite{Russell:10b}
\\
\\
\\
\\
\subsection{Tiefensuche}\index{Tiefensuche}
Die Tiefensuche gehört ebenfalls zu den uninformierten Suchalgorithmen.
Im Gegensatz zu der Breitensuche werden nicht die Ebenen nacheinander abgesucht sondern je Nachfolger angefangen 
beim Wurzelknoten werden bis sie keine weiteren Nachfolger mehr haben besucht. Erst dann wird der nächste Nachbar 
in der ersten Ebene besucht bis keine unbesuchten Knoten mehr vorhanden sind.Die Tiefensuche ist indirekt an vielen 
komplexeren Algorithmen beteiligt. Unter anderem kann die Tiefensuche auch für das Ermitteln von Zusammenhangskomponenten
oder für das Erzeugen eines Irrgartens verwendet werden.
\cite{Russell:10c}


\section{Dijkstra Algorithmus}
\label{Dijkstra Algorithmus}
\subsection{Definition}

Adeel Javaid beschrieb, dass „Der Dijkstra-Algorithmus (benannt nach seinem Entdecker E.W. Dijkstra) das Problem löst, den kürzesten Weg von einem Punkt in einem Diagramm (der Quelle) zu einem Ziel zu finden“\footnote{Dijkstra Algorithmus Definition \cite{Javaid2019}}.

\subsection{Prinzip}

Da nachgewiesen wurde, dass die kürzesten Wege von einem Quellknoten zu allen Orten in einem Diagramm tatsächlich in derselben Zeit gefunden werden können, wird dieses Problem manchmal als Problem der kürzesten Wege für eine einzige Quelle bezeichnet\cite{Javaid2019}.
\newline
\newline
Der Dijkstra-Algorithmus kann die beste Route wählen, die die Voraussetzungen für die Topologie-Roadmap erfüllt. Der klassische Dijkstra-Algorithmus unterteilt die Knoten im topologischen System in drei Gruppen: zunächst alle vorläufigen Knoten im System des Algorithmus, die nicht gekennzeichnet sind, und die Knoten, die sich während der Routenauswahl und der effizienten Routenauswahl kreuzen und verbinden sollen. Jede Screening-Schleife im optimalen Routenauswahlverfahren wählt den Knoten mit der kürzesten Pfadlänge vom momentanen Indikator knoten als permanenten Markierung-knoten, und der Dijkstra-Algorithmus iteriert, bis der aktuelle und der geplante oder alle Knoten erreicht sind. Der Knoten würde dann so lange bestehen bleiben, bis er durch einen permanenten Markierung-knoten ersetzt wird\cite{Zhou2013}.

\subsection{Dijkstras Vorteile}

\begin{itemize}
	\item Minhang Zhou und Nina Gao haben zitiert „Der Dijkstra-Algorithmus kann alle optimalen Pfade finden, und die Trefferquote dieser optimalen Pfade liegt bei 100 \%“\footnote{Ertster Vorteil von Dijkstra \cite{Zhou2013}}.
	\item Wenn der geplante Zielknoten erreicht ist, besucht der Dijkstra-Algorithmus die restlichen unerwünschten Knoten nicht\cite{Abusalim2020}. 
\end{itemize}


\subsection{Dijkstras Nachteile}

Der Hauptnachteil des Algorithmus besteht darin, dass er eine blinde Suche durchführt, wodurch viel Zeit und relevante Ressourcen verschwendet werden, oder anders ausgedrückt, er ist zeitintensiv. Ein weiterer Nachteil ist, dass er keine negativen Seiten verwalten kann, was zu azyklischen Graphen führt, und dass er häufig nicht die beste Route findet\cite{Mukhlif2020}.

\subsection{Dijkstras Pseudo-Code}

Der Dijkstra-Algorithmus arbeitet mit der Zuweisung einiger vorläufiger Entfernungswerte und versucht, diese schrittweise zu verbessern. Der Pseudocode des Algorithmus ist in der folgenden Abbildung dargestellt\cite{Huang2012}.
 \begin{figure}[H]
	\centering
	\includegraphics[width=1.0\textwidth]{images/Dijkstra_pseudoCode.PNG}
	\caption{\label{fig:Dijkstra}Dijkstra Algorithmus Pseudo-Code \cite{Abusalim2020}.}
\end{figure}

Abusalim Samah, Ibrahim Rosziati, Saringat Mohd, Jamel Sapiee und Wahab Jahari haben  diesen Pseudocode wie folgt beschrieben:
\newline
\newline
 „Beim Dijkstra-Algorithmus ist der Pfad nicht bekannt. Die Knoten werden in zwei Gruppen unterteilt: temporäre (t) und permanente (p).
\begin{itemize}
	\item  Zunächst wird die Entfernung des Quellknotens mit dem Wert Null initialisiert [distance(a) = 0], und die Entfernung der anderen Knoten wird mit dem Wert Unendlich belegt [distance(x) = infinity]. 
	\item Schritt 2: Suche nach dem Knoten x mit dem kleinsten Wert von d(x). Wenn es keine temporären Knoten gibt oder der Wert von d(x) gleich unendlich ist, wurde der Knoten x als permanent eingestuft, was bedeutet, dass sich d(x) und der übergeordnete Wert von d(x) nicht mehr ändern werden. 
	\item Schritt 3: Wenden Sie den folgenden Vergleich für jeden temporären Knoten mit der Bezeichnung vertex y an, der an x angrenzt“\footnote{Ausführliche Beschreibung von Dijkstra Pseudo Code \cite{Abusalim2020}}.
\end{itemize}


\section{Bellman-Ford-Algorithmus}
\label{Bellman-Ford-Algorithmus}
\subsection{Definition}

Vaibhavi Patel und Prof.ChitraBaggar haben benotet dass, „Der Bellman-Ford-Algorithmus verwendet Entspannung, um kürzeste Pfade auf gerichteten Graphen zu finden, die nur eine Quelle haben“\footnote{Bellman-Ford-Algorithmusn \cite{Vaibhavi2014}}.
\subsection{Prinzip}
Wenn es negative Gewichtsnusszyklen gibt, wird der Algorithmus sie erkennen. Eine negative Länge gibt es nicht, wenn es sich um Bereiche auf einer Karte handelt. Der Ballmann-Ford-Algorithmus ist im Allgemeinen mit dem Dijkstra-Algorithmus vergleichbar. Er entspannt alle Kanten |V| mal, wobei |V| die Menge der Scheitelpunkte ist\cite{Vaibhavi2014}.

\subsection{Bellman-Ford Pseudo-Code}\index{Bellman-Ford}
Der Bellman-Ford-Algorithmus wird wie in der Abbildung unten dargestellt ausgeführt.


 \begin{figure}[H]
	\centering
	\includegraphics[width=1.0\textwidth]{images/Bellman-Ford-Algorithmus_Pseudo-Code.PNG}
	\caption{\label{fig:Bellman}Bellman-Ford Pseudo-Code\cite{Abusalim2020}.}
\end{figure}
Abusalim Samah, Ibrahim Rosziati, Saringat Mohd, Jamel Sapiee und Wahab Jahari haben sie diesen Pseudocode wie folgt beschrieben:
\newline
\newline
\begin{itemize}
	\item   „Schritt 1: Setzen Sie den Abstand des Quellknotens s auf den Wert Null (distance[s] = 0) und weisen Sie den anderen Knoten einen Abstand von INFINITY zu. 
	\item Schritt 2: entspannt jede Kante (n - 1) Mal, wenn n die Anzahl der Knoten ist. Das Entspannen einer Kante bedeutet zu prüfen ob es möglich ist, den Weg zu dem Knoten, auf den die Kante zeigt, zu verkürzen, und, wenn ja, den Weg zu den Knoten durch die gefundene Route. 
	\item Schritt 3: Prüfen, ob der Graph einen negativen Zyklus hat, mit Ausführung der N-ten Schleife\footnote{Ausführliche Beschreibung von Bellman-Ford Pseudo Code“ \cite{Abusalim2020}}.
\end{itemize}








\chapter{Optimierungsstrategien}
\label{Optimierungsstrategien}

Die Anforderungen an die Laufzeit von Algorithmen zur Pfadplanung steigt auf Grund der immer komplexer werdenden Systemen stetig weiter an. In diesem Kapitel wird beschrieben, wie die bisher vorgestellten Algorithmen zur Pfadplanung für bestimmte Anwendungsgebiete optimiert werden können.

\section{Bidirektionale Suche}

Bei der bidirektionalen Suche wird ein Suchalgorithmus simultan aus zwei Richtungen laufen gelassen – vom Startknoten zum Zielknoten und umgekehrt. Der Suchalgorithmus wird bei diesem Vorgehen so modifiziert, dass die Abbruchbedingung dann eintritt, wenn beide Suchen denselben Knoten expandieren. Somit betrachtet jede der beiden Suchen nur die Hälfte des Graphen, was in einer Reduktion der Zeitkomplexität resultiert \cite{Russell2010}. Es ist jedoch zu beachten, dass die Platzkomplexität stark ansteigt, da in beiden Suchen eigene Priority-Queues verwaltet werden müssen. Außerdem ist es für viele Problemstellungen keinesfalls trivial, eine Suche rückwärts durchzuführen, da eine Methode zur Berechnung des Vorgängers eines Knotens gegeben sein muss \cite{Russell2010}.

%Eine Implementierung der Bidirektionalen Suche ist der Bidirektionale Dijkstra Algorithmus von Vaira und Kurasova. [7] Hier wird der in Abschnitt \ref{Algorithmen zur Pfadplanung} vorgestellte Dijkstra Algorithmus so modifiziert, dass eine bidirektionale Suche von zwei Prozessen auf einem Mehrkernprozessor parallelisiert durchgeführt wird. In mehreren experimentellen Analysen konnte gezeigt werden, dass die Laufzeit des parallelen bidirektionalen Dijkstra Algorithmus bis zu drei Mal kleiner ist als die des Standard Dijkstra Algorithmus.%

\section{Informed Search}
Ein Ansatz, um effizienter Lösungen für das Shortest Path Problem zu finden, ist die Informed Search Strategie, bei der problemspezifisches Wissen, das über die Definition des Problems hinausgeht, bei der Lösungsfindung berücksichtigt wird. Der nächste zu expandierende Knoten auf dem Pfad zum Zielknoten wird auf Basis einer Bewertungsfunktion $f(n)$ ausgewählt. Eine Komponente dieser Bewertungsfunktion ist eine heuristische Funktion $h(n)$, die die zu erwartenden Kosten des optimalen Pfades vom Knoten $n$ zum Zielknoten berechnet \cite{Russel2010}. Im Falle des Straßennetzes könnte hierzu die Länge der Luftlinie zwischen dem Knoten $n$ und dem Zielknoten verwendet werden \cite{Hart1968}.

Die einfachste Umsetzung dieser Strategie ist, nur die heuristische Funktion bei der Bewertung von Knoten heranzuziehen, sodass $f(n)=h(n)$ gilt. Dieses Vorgehen wird auch Greedy Best-First Suche genannt, da in jedem Schritt versucht wird, so nahe wie möglich an den Zielknoten zu gelangen \cite{Russel2010}. Auf diese Weise werden die Suchkosten, also die Anzahl der expandierten Knoten zwar minimiert, es kann jedoch nicht garantiert werden, dass die gefundene Lösung optimal ist \cite{Russel2010}.

Eine elaboriertere Umsetzung der Informed Search Strategie ist der A* Algorithmus zur Berechnung des kürzesten Pfades zwischen zwei Knoten \cite{Russel2010}. A* basiert auf Dijkstra’s Algorithmus und erweitert diesen um eine heuristische Funktion, um die Laufzeit zu reduzieren \cite{Peralta2020}. Die Bewertungsfunktion $f(n)$ für den A*-Algorithmus setzt sich zusammen aus den Kosten des optimalen Pfades vom Startknoten bis zum Knoten $n$, $g(n)$ und einer heuristischen Funktion $h(n)$, sodass gilt:
\begin{equation} \label{eq:3.1}
	f(n)=g(n)+h(n)
\end{equation}

Da verschiedene Heuristiken zur Konstruktion von $h(n)$ gewählt werden können, stellt A* streng genommen eine Familie von Algorithmen dar, wobei die Wahl einer Funktion $h(n)$ einen spezifischen Algorithmus der Familie selektiert \cite{Hart1968}. 

Hart, Nilsson und Raphael, die in \cite{Hart1968} die A*-Suche 1968 zum ersten Mal beschrieben haben, konnten nachweisen, dass A* vollständig und optimal ist, wenn die gewählte Heuristik zulässig und konsistent ist. Das heißt, dass unter den angegebenen Voraussetzungen für die Heuristik, immer ein Pfad vom Start- zum Zielknoten gefunden wird (sofern dieser existiert) und dass dieser Pfad in jedem Fall optimal ist. Des Weiteren konnte gezeigt werden, dass A* optimal effizient ist – es kann also keinen anderen optimalen Algorithmus geben, der garantiert weniger Knoten expandiert als A* \cite{Russel2010}.

\section{Preprocessing}
Eine weiter Optimierungsstrategie ist das Preprocessing, also die Vorverarbeitung des Graphen. Dabei wird häufig gefordert, dass die Platzkomplexität der vorverarbeiteten Daten linear in der Größe des zu bearbeitenden Graphen ist, da man in der Realität oft mit sehr großen Graphen arbeitet \cite{Goldberg2005}. 

\subsection{ALT-Algorithmen}
Eine Familie von Algorithmen, die auf Preprocessing basieren, sind die von Goldberg und Harrleson in \cite{Goldberg2005} vorgestellten ALT-Algorithmen. ALT ist ein Akronym für A* search, \textit{Landmarks} und \textit{Triangle Inequality}\footnote{Dreiecksungleichung}. 

In der Preprocessing-Phase des Algorithmus, wird eine kleine (konstante) Anzahl von Landmarken im Graphen ausgewählt, von denen aus dem kürzesten Pfad zu allen anderen Knoten bestimmt wird. Die so bestimmte Distanz wird in Kombination mit der Dreiecksungleich als Heuristik für die Bewertungsfunktion der A* Suche verwendet. 
\newpage
Die Dreiecksungleich besagt in diesem Fall, dass die Distanz zwischen einem Knoten $n$ und einem Zielknoten $s$ in jedem Fall größer oder gleich der Differenz  der Distanz zwischen $n$ und einer Landmarke $l$ und der Distanz zwischen $s$ und der Landmarke $l$ ist.

\begin{equation} \label{eq:3.2}
	dist(n,s) \ge dist(l,n)-dist(l,s)
\end{equation}


Somit stellt diese Differenz eine untere Schranke für die Distanz zwischen n und s dar und kann somit als Heuristik verwendet werden \cite{Goldberg2005}.

\subsection{Reach-Based Pruning}
Reach-Based Pruning \footnote{englischer Ausdruck für das Beschneiden von Bäumen. In der Informatik wird \textit{Pruning} oft als Ausdruck für das Vereinfachen von Graphen verwendet \cite{Wikipedia:00}.} ist eine von Gutman in \cite{Gutman2004} vorgestellte Preprocessing-Strategie zur Vereinfachung von Graphen. Der \textit{Reach} $r$ ist hierbei eine Metrik für einen Graphen $G$, die als Modifikation für den Dijkstra Algorithmus verwendet werden kann. 

Sei $P$ ein Pfad in $G$ von einem Startknoten $s$ zu einem Zielknoten $t$ und $v$ ein Knoten auf diesem Pfad $P$. Sei zudem $dist(v,w,P)$ die Distanz zwischen den Knoten $v$ und $w$ auf dem Pfad $P$. Dann ist

\begin{equation} \label{eq:3.3}
	r(v,P) = \min{(dist(s,v,P), dist(v,t,P))}
\end{equation}

der \textit{Reach} von $v$ auf $P$. Zudem ist der \textit{Reach} von $v$ in $G$, $r(v,G)$ definiert als das Maximum aller $r(v,Q)$ für alle kürzesten Pfade $Q$ in $G$. 

Gutman konnte beweisen, dass ein Knoten $v$ nur dann vom Dijkstra Algorithmus betrachtet werden muss, wenn 
\begin{equation} \label{eq:3.4}
	r(v,G) \ge \underline{dist(s,v)} \: \: \: \vee \: \: \: r(v,G) \ge \underline{dist(v,t)}
\end{equation}

gilt, wobei $\underline{dist(v,w)}$ eine untere Schranke für die Distanz zwischen zwei Knoten $v$ und $w$ darstellt. 

Die einfachste Möglichkeit die \textit{Reaches} aller Knoten zu bestimmen, ist alle kürzesten Pfade eines Graphen zu bestimmen und die Bedingung \ref{eq:3.4}  anzuwenden. Effizientere Vorgehen wurden in \cite{Goldberg2007} und \cite{Gutman2004} beschrieben.

\newpage
\section{Experimentelle Analyse}
Um die Auswirkung der hier vorgestellten Optimierungsstrategien zu veranschaulichen, wurde von Goldberg in \cite{Goldberg2007} die Laufzeit der folgenden Algorithmen verglichen:
\begin{itemize}
	\item \textit{\textbf{B}}: Bidirektionaler Dijkstra Algorithmus 
	\item \textit{\textbf{ALT}}: Algorithmus aus der ALT-Familie 
	\item \textit{\textbf{RE}}: Implementierung der Reach-Based Pruning Strategie 
	\item \textit{\textbf{REAL}}: Algorithmus mit zwei Preprocessing-Stufen (ALT und RE)
\end{itemize}
Als Input wurde das Straßennetz der San Francisco Bay Area mit $330 024$ Knoten und $793 681$ Kanten verwendet und jeder dieser Algorithmen wurde auf $10.000$ zufällig gewählten Paaren von Knoten angewendet. Dabei wurden die Laufzeit der Preprocssing-Phase und der Query-Phase der Algorithmen gemessen\footnote{Die Laufzeitmessungen wurden auf einem Toshiba Tecra 5 Laptop mit 2GB RAM und Dual-Core 2 GHz Processor durchgeführt} und in Tabelle \ref{tab:Experimentelle Analyse} dargestellt.

\begin{table}[h] % Beginn: Tabellen-Umgebung
	\centering % zentriert die Tabelle
	\def\arraystretch{1.5}
	\begin{tabular}{|c| c| c|}
		\hline % horizontale Linie
		\: \: \textbf{Algorithmus} \: \: & \: \: \textbf{Laufzeit Preprocessing} \: \: & \: \: \textbf{Laufzeit Query} \: \: \\
		\hline % horizontale Linie
		\textit{B} & - & $30,49$\,ms  \\
		%\hline % horizontale Linie
		\textit{ALT} & $5,7$\,s & $2,91$\,ms  \\
		%\hline % horizontale Linie
		\textit{RE} & $45,4$\,s & $0,55$\,ms  \\
		%\hline % horizontale Linie
		\textit{REAL} & $51,1$\,s & $0,28$\,ms  \\
		\hline % horizontale Linie
	\end{tabular}
	\caption{Messung der durchschnittlichen Laufzeit der Preprocessing- und Query-Phase der Algorithmen \textit{B}, \textit{ALT}, \textit{RE} und \textit{REAL} auf dem Straßennetz der San Francisco Bay Area.} % Tabellenunterschrift 
	\label{tab:Experimentelle Analyse} % ermöglicht Verweise
\end{table} % Ende: Tabellen-Umgebung

Man kann erkennen, dass die Laufzeit der Algorithmen mit Preprocessing-Phase (\textit{ALT}, \textit{RE}, \textit{REAL}) signifikant besser ist, als die der Algorithmen ohne Preprocessing-Phase (\textit{B}). Es ist jedoch zu beachten, dass die Laufzeit der Preprocessing-Phase um einige Größenordnungen höher ist, als die Laufzeit des eigentlichen Query-Algorithmus.




\chapter{Anwendungen}
\label{Anwendungen}

In diesem Kapitel wird beschrieben, wie Pfadplanung Algorithmen in realen Anwendungen eingesetzt werden können.
\section{GIS(Geoinformationssystem)}
\label{GIS(Geoinformationssystem)}
% \subsection{Definition}

Ein geografisches Informationssystem (GIS) ist ein System zum Erstellen, Verwalten, Analysieren und Kartieren verschiedener Arten von Daten. GIS verknüpft Daten mit einer Karte und integriert Standortdaten mit verschiedenen Arten von beschreibenden Daten. Dies ist die Grundlage für die Kartierung und Analyse in der Wissenschaft und in fast allen Branchen.
Bei diesen Daten handelt es sich um Informationen über Objekte auf der Erde wie Städte, Eisenbahnstrecken, Flüsse usw \cite{Vaibhavi2014}. 

% \subsection{Geoinformationssystem mit Dijkstra-Algorithmus}
Algorithmen für den kürzesten Weg werden in Kartenplattformen wie Google Maps verwendet, um den kürzesten Weg zwischen zwei Punkten zu ermitteln.
Obwohl der Dijkstra-Standardalgorithmus in diesem Fall anwendbar zu sein scheint, dauert das Routing des Pfades vom Startpunkt zum Endpunkt in einer großen Datenmenge sehr lange \cite{HamidAli2020}.

 Um bei der Berechnung des kürzesten Weges Zeit zu sparen und eine bessere Lösung zu erhalten. Der bessere Ansatz besteht darin, angelehnt an \cite{HamidAli2020}. Vor dem Start des Dijkstra-Algorithmus einen temporären Datensatz zu erstellen und ihn so zu behandeln, als wäre er der Graph, mit dem der Algorithmus arbeiten muss.
 Nachdem die Start- und Endknoten ermittelt wurden, wird dieser Datensatz erstellt. Der Datensatz kann anhand der Koordinaten des Start- und des Zielknotens aus den Hauptdaten ausgeschlossen werden, indem nur die Knoten ausgewählt werden, die sich innerhalb des von den beiden Knoten gebildeten reduzierten Suchbereichs befinden.
 


\newpage
\section{Mobile Roboter}
\label{Mobile Roboter}
Die Steuerung mobiler Roboter ist ein wichtiges Thema in intelligenten Städten und die Pfadplanung ist eine wichtige Komponente unter den mobilen Roboter-Technologien, die dem Roboter hilft, den Weg von einem Startpunkt zu einem Zielpunkt zu finden und dabei sicher und zuverlässig alle Hindernisse in einer statischen und dynamischen Umgebung während der Reise zu vermeiden \cite{Myung21,Hong-mei17}.

\subsection{Umgebungsmodell}
Das Rolling-Window-Prinzip wählt einen lokal optimalen Zielzustand innerhalb des Erfassungsbereichs der Sensoren, wenn eine mögliche Kollision vorhergesagt wird, während sich der Roboter auf dem Weg in einer unbekannten Umgebung bewegt, während die Algorithmen Dijkstra und A* in erster Linie für die Pfadplanung in einer bekannten Umgebung eingesetzt werden.
\newline
Da sowohl der Dijkstra- als auch der A*-Algorithmus Gitter-basierte Suchmethoden sind, sollte zunächst das Gittermodell der umgebenden Karte erstellt werden. Bei der Gittermethode wird die Karte in benachbarte Gitter gleicher Größe unterteilt. 
Die Größe des mobilen Roboters bestimmt die Größe des Gitters und beeinflusst die Suchgenauigkeit und Effizienz des Algorithmus \cite{Hong-mei17}.

\subsection{Dijkstras Algorithmus und A*-Algorithmus}

Die Idee ist, A*- und Dijkstra-Algorithmen miteinander zu kombinieren, um die Effizienz der Planung einer idealen kollisionsfreien Route für einen mobilen Roboter zu erhöhen. Bei der Neuplanung wird die Dijkstra-Methode zur Vorverarbeitung der statischen Umgebungskartendaten angewendet, die optimale Routen vom Ziel zu allen freien Zuständen plant und speichert. Wenn der Roboter auf seinem Weg zum Ziel auf ein bewegliches Hindernis trifft, wählt das Rolling-Window-Prinzip einen lokal optimalen Zustand als nächsten Zielzustand. Dann wird mit dem A*-Algorithmus eine lokal optimale Route von der Roboterposition zum lokalen Ziel neu geplant.
Die neue Route kann den Roboter um Hindernisse herumführen \cite{Hong-mei17}.



\section{Autonome Navigation}
\label{Autonome Navigation}

Pfadsuchalgorithmen werden ebenfalls im Bereich des Autonomen Fahrens, oder auch bei unbemannten Flugfahrzeugen verwendet
um sichere, effiziente, kollisionsfreie und kostengünstige Wege von Start zum Ziel zu führen, was die Wahl des richtigen Pfadsuchalgorithmen
zu einer wichtigen Aufgabe macht. Es hängt unter anderen Faktoren die Geometrie des Fahrzeugs von dieser Wahl ab \cite{Karur:21}.
Mit der zunehmenden Verbreitung von autonomen Fahrzeugen, die immer mehr Wegfindung und -planung erfordert, sind Pfadsuchalgorithmen
zu einem neuen Schwerpunkt der autonomen Steuerung geworden \cite{Karur:21}.
Da mobile Roboter in vielen Anwendungen eingesetzt werden, haben Forscher Methoden entwickelt, um die 
Anforderungen an mobile Roboter effektiv erfüllen zu können und einige Herausforderungen für die Umsetzung einer vollständig oder
teilweise autonomen Navigation in unübersichtlichen Umgebungen zu bewältigen \cite{Karur:21}.
So wird die Wahl des richtigen Pfadplanungsalgorithmus von der kinematischen Bewegungsgestaltung des Roboters/Fahrzeugs,
den zur Verfügung stehenden Rechenressourcen, sowie der sensorischen Ausstattung des Fahrzeugs bestimmt \cite{Karur:21}.

Die Leistung und Komplexität des verwendeten Algorithmus hängt auch vom Anwendungsfall ab \cite{Karur:21}.

Somit gibt es nicht \emph{den perfekten Pfadsuchalgorithmus für Autonomes Fahren}, aber es finden viele verschiedene Algorithmen eine 
Anwendung in der autonomen Navigation.
\chapter{Zusammenfassung und Ausblick}
\label{Zusammenfassung und Ausblick}
In den letzten Jahren hat die Pfadplanung\index{Pfadplanung} immer mehr an Relevanz gewonnen, zum Beispiel im Bereich des autonomen Autofahrens wird immer mehr Forschung 
im Bereich der Pfadplanungsalgorithmen\index{Pfadplanungsalgorithmen} betrieben. \cite{Karur:21}

In diesem Paper haben wir aufeinander aufbauend Algorithmen\index{Algorithmen} vorgestellt, die in einem Programm mit bestimmten Eingaben und Umgebungen 
wie z.B Graphen verwendet werden können um von einem gegeben Start ein oder mehrere Ziele zu finden und dabei durch verschiedene Bewertungskriterien\index{Bewertungskriterien} den besten Weg zu finden.
Mit diesen Algorithmen kann ein Programm durch eine Sequenz von Aktionen sein Ziel erreichen. Diesen Prozess nennt man Suche.
\cite{Russell:10c}
\begin{itemize}
    \item Bevor das Programm mit der Suche nach der besten Lösung beginnen kann, muss erst ein Ziel identifiziert und das Problem genau definiert werden.
    \item Suchalgorithmen behandeln Zustände und Aktionen atomar: Sie berücksichtigen keine interne Struktur, die sie besitzen könnten.
    \item Ein allgemeiner uninformierter\index{uninformiert} TREE-SEARCH-Algorithmus berücksichtigt alle möglichen Wege, um eine Lösung zu finden, während ein informierter GRAPH-SEARCH-Algorithmus redundante Wege vermeidet.
    \item Uninformierte Pfadsuchalgorithmen haben nur Zugriff auf die Problemdefinition. Die grundlegenden Algorithmen sind wie folgt:
    \begin{itemize}
        \item Die Breadth-first-Suche expandiert zuerst die flachsten Knoten; sie ist vollständig, optimal für einheitliche Pfadkosten, hat aber eine exponentielle Raumkomplexität.
        \item Die Tiefensuche\index{Tiefensuche} expandiert zuerst den tiefsten nicht expandierten Knoten. Sie ist weder vollständig noch optimal, hat aber eine lineare Raumkomplexität.
        \item Die iterative Vertiefungssuche ist eine Wiederholung der Tiefensuche mit zunehmender Tiefenbegrenzung, bis ein Ziel gefunden wird. Sie ist vollständig, optimal für die Kosten pro Schritt, hat eine vergleichbare Zeitkomplexität\index{Zeitkomplexität} wie die Breitensuche\index{Breitensuche} und eine lineare Raumkomplexität\index{Raumkomplexität}.
        \item Der Greedy Dijkstra-Algorithmus\index{Dijkstra} der zwar bei der blinden Suche Zeit vergeudet aber dafür optimal ist und eine Trefferquote von 100\% hat.
        \item Die Optimierung\index{Optimierung} durch eine bidirektionale Suche\index{bidirektionale Suche} kann die Zeitkomplexität enorm reduzieren, ist aber nicht immer anwendbar und kann zu viel Speicherplatz beanspruchen.
    \end{itemize}
    \item Informierte\index{informiert} Suchmethoden können auf heuristische Funktionen zurückgreifen, die die Kosten einer Lösung schätzen
    \begin{itemize}
        \item Der generische Best-First-Suchalgorithmus\index{Best-First} wählt einen Knoten für die Expansion gemäß einer Bewertungsfunktion aus.
        \item Der Greedy best-first search expandiert Knoten mit minimalem heuristischem\index{Heuristik} Funktionswert. Er ist nicht optimal, aber oft effizient.
        \item A*-Suche\index{A*} expandiert Knoten mit minimalem Heuristischen Funktions- und Pfadkostenwerten.\index{Pfadkosten} A* ist vollständig und optimal, vorausgesetzt die heuristische Funktion ist zulässig.
        \item ALT-Algorithmen\index{ALT}, die aufgebaut auf A* durch Preprocessing noch optimiertere Ergebnisse erzeugt.
        \item Reach-based Pruning\index{Reach based Pruning}, welches den Dijkstra-Algorithmus um eine Metrik erweitert und dadurch optimiert.
    \end{itemize}
\end{itemize}
\cite{Russell:10c}

Alle diese Algorithmen haben diverse Vor- und Nachteile und daraus ergeben sich verschiedene 
Anwendungsbereiche\index{Anwendungsbereiche} auf welche wir in Kapitel \ref*{Anwendungen} gesondert eingegangen sind.
% ...
%--------------------------------------------------------------------------
\backmatter                        		% Anhang
%-------------------------------------------------------------------------
\bibliographystyle{geralpha}			% Literaturverzeichnis
\bibliography{literatur}     			% BibTeX-File literatur.bib
%--------------------------------------------------------------------------
\printindex 							% Index (optional)
%--------------------------------------------------------------------------
\begin{appendix}						% Anhänge sind i.d.R. optional
   %\include{chapters/Glossar}			% Glossar   
   \chapter*{Arbeitsverteilung}

\textbf{Teilnehmer 1: Mohammed Salih Mezraoui} 
\\
 Inhalte:
\\
\\
\\
\\
\textbf{Teilnehmer 2: David Gruber} 
\\
Inhalte: 
\begin{itemize}
    \item Kapitel \ref{Einleitung und Problemstellung}: Einleitung und Problemstellung
    \item Kapitel \ref{Uninformierter Ansatz}: Uninformierter Ansatz
    \item Kapitel \ref{Autonome Navigation}: Autonome Navigation
    \item Kapitel \ref{Zusammenfassung und Ausblick}: Zusammenfassung und Ausblick
\end{itemize}
\paragraph*{}
\textbf{Teilnehmer 3: Marius Müller} 
\\
 Inhalte: 
\begin{itemize}
	\item Kapitel 3: Optimierungsstrategien
\end{itemize}		% Welcher Studierende hat welche Texte formuliert?
   
\end{appendix}

\end{document}
