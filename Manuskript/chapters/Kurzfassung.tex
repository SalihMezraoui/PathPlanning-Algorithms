\kurzfassung

%% deutsch
\paragraph*{}
In dieser wissenschaftlichen Arbeit werden die wichtigsten Algorithmen zur Pfadplanung dargestellt und analysiert. Es wird anhand der Anwendung in Geoinformationssystemen und bei mobilen Robotern aufgezeigt, wie diese Algorithmen eingesetzt werden und welche Stärken und Schwächen damit einhergehen.  Da die Fragestellungen, die mit Algorithmen zur Pfadplanung bearbeitet werden, immer komplexer werden, steigen auch die Anforderungen an Zeit- und Platzkomplexität. Daher werden in dieser Arbeit die wichtigsten Optimierungsstrategien für Algorithmen zur Pfadplanung dargestellt. In einer experimentellen Analyse konnte gezeigt werden, dass die Laufzeit des Algorithmus von Dijkstra durch den Einsatz von Preprocessing und heuristischer Funktionen um mehrere Größenordnungen gesenkt werden kann. 

%% englisch
\paragraph*{}
In this scientific work, the most important algorithms for path planning are presented and analysed. The application in geoinformation systems and mobile robots shows how these algorithms are used and what strengths and weaknesses are associated with them. As the issues addressed by path planning algorithms are becoming increasingly complex, the demands on time and space complexity are also increasing. Therefore, this paper presents the most important optimization strategies for path planning algorithms. An experimental analysis showed that the runtime of the Dijkstra algorithm can be reduced by several orders of magnitude by using preprocessing and heuristic functions.
