\chapter{Algorithmen zur Pfadplanung}\index{Algorithmen}\index{Pfadplanung}\index{Algorithmus}
In diesem Kapitel werden die Algorithmen \emph{Dijkstra} und \emph{Bellman-Ford} zur Pfadplanung beschrieben.

\section{Einleitung}
\label{Was ist Pfadplanung?}
Die Pfadplanung ist ein nichtdeterministisches Problem mit polynomialer Laufzeit ("NP"), bei dem es darum geht, einen Pfad zu finden, 
der in einem System den Ausgangspunkt mit dem Ziel verbindet. Der zu wählende Weg (die beste Route) wird durch Beschränkungen 
und Bedingungen bestimmt \cite{Karur:21}.

In der Informatik, insbesondere in der Graphentheorie, ist das Problem des kürzesten Weges bekannt. Der kürzestmögliche Weg von 
einer Quelle zu einem Ziel hat die geringsten Längenanforderung.

Die Dijkstra- und Bellman-Algorithmen für den kürzesten Weg sind in einer Vielzahl von Bereichen und Anwendungen
weit verbreitet. Beispiele für diese Anwendungen sind Routing-Protokolle für Netzwerke, Routenplanung, 
Verkehrssteuerung, Pfadfindung in sozialen Netzwerken, Computerspiele und Navigationssysteme \cite{Panda:18}.

\section{Uninformierter Ansatz}
\label{Uninformierter Ansatz}
Als Einstieg in die Thematik der Pfadplanung wird der Uninformierte Ansatz beschrieben.
Uninformierte Algorithmen werden auch „blind“ genannt, weil bei ihrer 
Suche auf keine zusätzlichen Informationen (wie z.B. Gewichtungen) zurückgegriffen wird \cite[81]{Russell:10}.
\subsection{Breitensuche}
\label{Breitensuche}
Die Breitensuche gehört zu den uninformierten Suchalgorithmen. 
Der erste Schritt der Breitensuche ist es alle verbundenen Knoten ersten Grades des Wurzelknotens zu besuchen.
Dies wird Ebene für Ebene im Baum wiederholt, bis alle Knoten besucht wurden. 
Die Breitensuche findet weitestgehend in der Graphentheorie ihre Anwendung \cite[81]{Russell:10}.
\\
\\
\\
\\
\subsection{Tiefensuche}
\label{Tiefensuche}
Die Tiefensuche gehört ebenfalls zu den uninformierten Suchalgorithmen.
Das Vorgehen bei der Tiefensuche ist es zuerst alle Unterknoten eines direkten Nachbarn des Wurzelknotens zu besuchen bis dieser 
keine Unterknoten mehr hat.
Erst dann wird der nächste Nachbar in der ersten Ebene besucht bis keine unbesuchten Knoten mehr vorhanden sind \cite[85,86]{Russell:10}. 
Die Tiefensuche ist indirekt an vielen komplexeren Algorithmen beteiligt. 
So nutzt die iterative Vertiefungssuche die klassische Tiefensuche in Wiederholung in Verbindung mit einer inkrementellen 
Tiefenbegrenzung bis ein Ziel gefunden wird \cite[108,109]{Russell:10}.
Außerdem kann die Tiefensuche auch für das Ermitteln von Zusammenhangskomponenten
oder für das Erzeugen eines Irrgartens verwendet werden \cite[85,86]{Russell:10}.

\section{Dijkstra Algorithmus}
\label{Dijkstra Algorithmus}

Der Dijkstra-Algorithmus (benannt nach seinem Entdecker E.W. Dijkstra) ist ein bekannter Algorithmus auf dem Gebiet der optimalen
Pfadwahl und wird verwendet, um den kürzesten Pfad von einem Startpunkt in einem Graphen zu einem Zielpunkt zu finden \cite{Javaid2019}.

\subsection{Funktionsprinzip}

Das Grundkonzept des Algorithmus ist wie folgt:
\\ \\
In einem ersten Schritt wird der Startknoten festgelegt und der Abstand zwischen ihm und den anderen Knoten des Graphen berechnet. 
Gibt es keine Kante, die diesen Knoten mit dem Startpunkt verbindet, ist der Abstand unendlich; gibt es eine Kante, die 
diesen Knoten mit dem Startpunkt verbindet, ist der Abstand $n$; und das niedrigste Gewicht (wenn es mehrere Kanten gibt) ist $n$.

Der zweite Schritt besteht darin, den Punkt mit dem kürzesten Abstand zum Startpunkt zu ermitteln und zu speichern. Die zuvor ermittelte 
Entfernung wird mit derjenigen verglichen, die über den soeben beiseite gelegten Punkt für alle verbleibenden Punkte 
ermittelt wurde, und nur der kleinste der beiden Werte wird beibehalten. Dieser Vorgang wird so lange wiederholt, bis es keine Punkte
mehr gibt oder bis der Ankunftspunkt gewählt ist \cite{Zhou:19}.
\\ \\
Im Folgenden wird die Umsetzung des unten stehenden Pseudocodes  \ref{Dijkstra Algorithmus Pseudocode} erläutert, angelehnt an \cite{Abusalim2020}.

Beim Dijkstra-Algorithmus ist die Route unbekannt. Die Knoten werden als temporär $(t)$ oder permanent $(p)$ eingestuft.
Ein temporäres Label wird geändert, wenn eine kürzere Route zu einem Knoten gefunden werden kann. Wenn keine bessere Route gefunden werden kann, wird der Status des temporären Labels in permanent geändert.
\begin{itemize}
	\item Schritt1: die Entfernung des Quellknotens wird der Wert Null zugewiesen $[distance (source) = 0]$, und die Entfernung der anderen
		Knoten wird auf unendlich gesetzt $[distance(x) = Infinity]$.
	\item Schritt 2: Suche des Knotens $x$ mit der kürzesten Entfernung $d(x)$. Wenn $d(x)$ unendlich ist oder es keine temporären Knoten gibt,
		wurde der Knoten $x$ als permanent markiert, was bedeutet, dass sich $d(x)$ und sein übergeordneter Wert nicht ändern werden.
	\item Schritt 3: Für jeden temporären Knoten mit dem Label $y$, der an $x$ angrenzt, wird der folgende Vergleich durchgeführt:

	
\end{itemize}
% \begin{gather*} \label{eq:2.1}
wenn
\begin{equation} \label{eq:2.1}
	d(x) + w (x, y) < d(y)	
\end{equation}
dann
\begin{equation} \label{eq:2.2}
	D(y) = d(x) + w (x, y)
\end{equation}
\newline
Wenn die Entfernung des gekennzeichneten Knotens $d(y)$ größer ist als die Entfernung des gekennzeichneten Knotens $d(x)$ plus Verbindungsgewicht $w(x, y)$, 
wird die Entfernung des gekennzeichneten Knotens $y$ gemäß den Gleichungen \ref{eq:2.1} und \ref{eq:2.2} aktualisiert in $D(y)$.

\noindent \\
\begin{minipage}{1.0\textwidth} \small
\begin{lstlisting}
$\textbf{function}$ DIJKSTRA (Graph, source)
	create vertex set D
	for each vertex v in Graph:
		distance[v] $\leftarrow$ INFINITY
		previous[v] $\leftarrow$ UNDEFINED
		add v to D
	distance[source] $\leftarrow$ 0
	
	while D is not empty do:
	    u $\leftarrow$ in D with min distance[u]
	    remove u from D
	    for each neighbour v of u:
	          alt  $\leftarrow$ distance[u] + length(u,v)
	          if alt < distance[v]
	              distance[v]  $\leftarrow$  alt
	              previous[v]  $\leftarrow$  u
	end while
	return distance[], previous[]
end function

\end{lstlisting}
\captionof{lstlisting}{Dijkstra-Algorithmus Pseudocode, angelehnt an \cite{Abusalim2020}.}
 \label{Dijkstra Algorithmus Pseudocode}
\end{minipage}


\subsection{Vor- und Nachteile}
Der Dijkstra-Algorithmus hat zwei wesentliche Vorteile: Er kann alle optimalen Pfade finden und die Trefferquote dafür liegt
bei 100\%. Der zweite Vorteil ist, dass er die verbleibenden unerwünschten Knoten nicht besucht, wenn der beabsichtigte Zielknoten erreicht
ist \cite{Zhou:19,Abusalim2020}.
\\ \\
Der Hauptnachteil des Algorithmus besteht darin, dass er eine blinde Suche durchführt, was eine erhebliche Menge an Zeit und Ressourcen vergeudet. 
Ein weiterer Nachteil ist, dass er nicht mit negativen Kanten umgehen kann, was zu azyklischen Graphen führt, und daher nicht immer den kürzesten 
Weg findet \cite{Mukhlif2020}.


\section{Bellman-Ford-Algorithmus}
\label{Bellman-Ford-Algorithmus}

In 1958 veröffentlichte Richard Bellman den Bellman-Ford-Algorithmus, einen Graphen-Suchalgorithmus zur Ermittlung des kürzesten 
Pfades \cite{Abusalim2020,Sulaiman18}.
\subsection{Eigenschaften}
Um kürzeste Wege auf gerichteten Graphen zu finden, nutzt der Bellman-Ford-Algorithmus die Relaxation \cite{Vaibhavi2014}.
Relaxation bedeutet, dass geprüft wird, ob es möglich ist, den Weg zu dem Knoten, auf den die Kante zeigt, zu verkürzen, 
und wenn dies der Fall ist, wird der Weg zum Knoten durch den entdeckten Weg ersetzt \cite{Abusalim2020}.
Der Bellman-Ford Algorithmus kann auch mit negative Kantengewichten umgehen. 
Wenn es Zyklen mit negativem Gewicht gibt, wird der Algorithmus sie erkennen (so dass es keine Lösung gibt) \cite{Vaibhavi2014}.

Die Vorteile dieses Algorithmus sind unter anderem, dass es sich um einen dynamischen Algorithmus handelt, dass er negative gerichtete 
Kanten (und auch positive) berechnen kann, dass er die Kosten für den Aufbau des Netzes reduzieren kann, indem er den kürzesten Weg von einem 
Knoten zum anderen findet, und dass er die Anzahl der aufgebauten Router-Pfad reduzieren kann \cite{Abusalim2020}.

\subsection{Funktionsprinzip}\index{Bellman-Ford}

Angelehnt an \cite{Abusalim2020} wird der Bellman-Ford-Algorithmus wie folgt ausgeführt:
\\
\begin{itemize}
	\item Schritt 1: Zuweisung des Abstandswertes des Startpunktes $s$ zu null $(distance[s] = 0)$ und des Abstandswertes der anderen Punkte zu 
		$INFINITY$.
	\item Schritt 2: Wenn $n$ die Anzahl der Knoten ist, wird jede Kante $(n - 1)$ Mal relaxiert. 
	\item Schritt 3: Mit der N-ten Schleife wird geprüft, ob der Graph negative Zyklen aufweist.
\end{itemize}
Der Bellman-Ford-Algorithmus wird wie in Listing  \ref{lst:Bellman-Ford Pseudo-code} unten dargestellt ausgeführt.

\noindent \\
\begin{minipage}{1.0\textwidth} \small
\begin{lstlisting}[label=lst:Bellman-Ford]
$\textbf{function}$ bellmanFord (G, s)
	for each vertex V in G:
		distance[v] $\leftarrow$ INFINITY
		previous[v] $\leftarrow$ NULL
	distance[s] $\leftarrow$ 0
	for each vertex V in G
	    for each edge(u,v) in G
	         alt  $\leftarrow$ distance[u] + length(u,v)
	         if alt < distance[v]
	             distance[v] $\leftarrow$ alt
	             previous[v] $\leftarrow$ u
	 
	for each edge(u,v) in G
	       if distance[u] + length(u,v) < distance[v]
	               Error: Negative Cycle exists
return distance[], previous[]
end function
       
\end{lstlisting}
\captionof{lstlisting}{Bellman-Ford-Algorithmus Pseudocode, angelehnt an \cite{Abusalim2020}.}
\label{lst:Bellman-Ford Pseudo-code}
\end{minipage}







