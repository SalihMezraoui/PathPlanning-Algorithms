\documentclass[a4paper,12pt]{book}

\usepackage[a4paper, inner=1.7cm, outer=2.7cm, top=1.5cm, bottom=2cm, bindingoffset=1.2cm]{geometry}
\usepackage[german]{babel}
\usepackage{microtype}
\usepackage{fancyhdr}


\pagestyle{fancy}
\fancyfoot{}
\fancyfoot[C]{\thepage}

\begin{document}

\chapter*{Review zur Ausarbeitung der Gruppe 7:}
\section*{Allgemeine Bewertung / Allgemeiner Eindruck}
Ich finde gut, dass ihr Quellen in der Einleitung verwendet habt, auch wenn mehrmals falsch zitiert wurde. 
Der Abschnitt mit den Optimierungsstrategien hat mir sehr gut gefallen.
Aber bei mir entsteht der Eindruck, dass ihr nicht zusammen gearbeitet und euch nicht gegenseitig korrigiert habt. Dies fällt besonders auf wie ihr im Text zitiert. 
\\ \\
\section*{Anmerkungen}
\textbf{Anmerkung 1 (Deckblatt):}
Ich vermute bei der Gruppe soll nur die Nummer eingetragen werden also \glqq 7\grqq.
Die Abgabe wurde auf den 17.07.2022 verschoben.
\\ \\
\textbf{Stellungnahme zu 1:} 
Gruppennummer auf 7 geändert und Datum korrigiert.
\\ \\
\textbf{Anmerkung 2 (Kurzfassung):} 
\begin{itemize}
	\item Eine Englische Kurzfassung sollen wir nicht schreiben, siehe Seite 25 im Script 00-WAL-Informationen. 
	\item Wer hat die Kurzfassung geschrieben? In der Arbeitsverteilung ist diese nicht aufgelistet.
	\item Die \glqq wichtigsten\grqq{} Algorithmen würde ich umformulieren, da es ein wertender Begriff ist.
	\item \glqq, wie diese Algorithmen eingesetzt werden...\grqq, welche Algorithmen werden eingesetzt?
	\item \glqq Da die Fragestellungen, ...\grqq, welche Fragestellungen werden in eurer Arbeit behandelt? 
\end{itemize}
\leavevmode \\
\textbf{Anmerkung 3 (Inhaltsverzeichnis):}
Meiner Meinung nach wurde die Ausarbeitung, für 3000 Wörter, zu weit untergliedert. 
In 2.1 würde ich die Überschrift umformulieren, laut Script 01-WissSchreiben Seite 38 sollen wir keine Fragen verwenden.
Die Abschnitte 2.3.1 und 2.4.1 sind für mich einleitende Worte, daher würde ich einfach beide Überschriften entfernen. Außerdem ist die Struktur bei Dijkstra und Bellman-Ford nicht einheitlich. Vor- und Nachteile könnten zu einer Überschrift zusammengefasst werden und sollten dann bei beiden Algorithmen vorkommen. Bei 2.3 und 2.4 wurde einmal ein Bindestrich und einmal kein Bindestrich verwendet, da würde ich ein einheitliches Schema verwenden. Bei 2.3.2 und 2.4.2 würde ich das Prinzip mit Funktionsprinzip ersetzen. Außerdem würde ich bei einem Unterpunkt nicht nochmal zusätzlich Dijkstras.. schreiben, das wirkt redundant.  
\\\\
\textbf{Anmerkung 4 (Umgangssprache und unpräzise Begriffe):}
Es wurden zu viele unpräzise Begriffe und Umgangssprache verwendet. Begriffe wie \glqq man, oft, häufig, viele, wir\grqq{} sollten umformuliert beziehungsweise gestrichen werden. Zum Beispiel anstatt \glqq viele Bereiche\grqq{}, in der Einleitung, zu schreiben könnte man die einzelnen Bereiche nennen. 
\\ \\
\textbf{Anmerkung 5 (Quellen):}
Ich bin mir nicht sicher ob die Art wie ihr zitiert so gewünscht ist. Herr Hepke hat uns in der Vorlesung gezeigt, dass wenn man mehrmals aus einem Buch zitiert dieses nur einmal im Literaturverzeichnis vorkommen soll. Im Text würde man dann zum Beispiel  \glqq [RN10, 92-102]\grqq{} schreiben. Außerdem sind Fußnoten nicht für Quellen geeignet. Wenn von einer Quelle mittels Anführungszeichen zitiert wird, dann ist der Text eins zu eins aus der Quelle zu übernehmen. Paraphrasierungen sind ohne Anführungszeichen zu schreiben. 
\\ \\
\textbf{Anmerkung 6 (Abbildungen):}
Abbildungen sollten in Anlehnung an XX, selbst erstellt werden. Zudem sollten die Texte in den Abbildungen übersetzt werden.  
Außerdem sollten die Abbildungen nicht farblich sein, es sei den Farbe wird dringend benötigt.
\\ \\
\textbf{Anmerkung 7 (Mohammed Salih Mezraouir):}
Du hast zu viel geschrieben, das gibt wahrscheinlich 2 Notenstufen Abzug. Fußnoten sind nicht für Quellen geeignet.
\\ \\
\textbf{Anmerkung 8 (David Gruber):}
Ich würde die Breiten und Tiefensuche weiter ausbauen. Dort könnte man Abbildungen einbauen und diese schrittweise erläutern. Momentan hast du etwas wenig an eigenem Text, weil du die Einleitung und die Zusammenfassung mit ca. 650 Wörtern geschrieben hast. Deine Zitierungen würde ich überarbeiten, eine Referenz erfolgt nicht nach einem Punkt. 
\\ \\
\textbf{Stellungnahme zu 8:}
Die Breiten und Tiefensuche auszubauen ist eine gute Idee. Bei der Wortanzahl wurde diese allerdings nicht berücksicht, genauso wie der Abschnitt 4.3 Autonome Navigation vermutlich nicht berücksicht wurde.
Ursprüngliche Wortanzahl waren nach meiner Zählung 976 Wörter.
\\ \\
\textbf{Anmerkung 9 (Einleitung und Problemstellung):}
Aus Wikipedia sollen wir nicht zitieren. Ihr beschreibt was man unter Pfadplanung versteht, aber was sind eigentlich Algorithmen?
\\ \\
\textbf{Stellungnahme zu 9:}
Satz und Quelle ersetzt.
\\ \\
\textbf{Anmerkung 1 (Einleitung und Problemstellung):}
\glqq In diesem Paper werden...\grqq, Quelle?
\\ \\
\textbf{Stellungnahme zu 1:}
Berechtigte Anmerkung. Quelle wurde hinzugefügt.
\\ \\
\textbf{Anmerkung 10 (Einleitung und Problemstellung, Seite 1):}
\glqq Ob im Bereich der Netzwerkroutenplanung oder bei KI-Spielern in Computerspielen Pfadsuchalgorithmen sind so relevant wie nie.\grqq, Quelle? 
\\ \\
\textbf{Stellungnahme zu 10:}
Quelle besser markiert.
\\ \\
\textbf{Anmerkung 11 (Einleitung und Problemstellung, Seite 1):}
Seid ihr euch sicher das der Bellman-Ford-Algorithmus \glqq intuitiv\grqq{} den kürzesten Weg liefert?
\\ \\
\textbf{Stellungnahme zu 11:}
zu \glqq nahezu intuitiv \grqq korrigiert
\\ \\
\textbf{Anmerkung 12 (Einleitung und Problemstellung, Seite 1):}
Wieso wird zweimal Google Maps beschrieben? 
\\ \\
\textbf{Stellungnahme zu 12:}
Zweite Erwähnung von Google Maps entfernt.
\\ \\
\textbf{Anmerkung 13 (2 Algorithmen zur Pfadplanung, Seite 2):}
Wer hat die einleitenden Worte geschrieben?
\\ \\
\textbf{Anmerkung 14 (2 Algorithmen zur Pfadplanung, Seite 2):}
Wird nur der Algorithmus von Bellman oder der von Bellman-Ford beschrieben?
\\ \\
\textbf{Anmerkung 15 (2.2.1 Breitensuche, Seite 2):}
\grqq Bei der Breitensuche ... \glqq, diesen Satz würde ich umformulieren.
\\ \\
\textbf{Stellungnahme zu 15:}
Satz wurde besser umformuliert.
\\ \\
\textbf{Anmerkung 16 (2.2.2 Tiefensuche, Seite 3):}
\grqq Im Gegensatz zu der Breitensuche... \glqq, diesen Satz würde ich umformulieren.
\\ \\
\textbf{Stellungnahme zu 16:}
Satz wurde umformuliert.
\\ \\
\textbf{Anmerkung 17 (2.2.2 Tiefensuche, Seite 3):}
\grqq sind.Die\grqq, Leerzeichen fehlt.
\\ \\
\textbf{Stellungnahme zu 17:}
Leerzeichen hinzugefügt.
\\ \\
\textbf{Anmerkung 18 (2.3.1 Definition, Seite 3):}
Ich würde auf das original Paper verweisen. Außerdem schreibst du in deiner Definition, dass nur zu einem Ziel der Weg gefunden wird. In Prinzip schreibst du allerdings, dass zu allen Zielen der Weg gefunden wird. 
\\ \\
\textbf{Anmerkung 19 (2.3.2 Prinzip, Seite 3):}
Du schreibst das die Knoten in drei Kategorien unterteilt werden, es werden allerdings nur 2 Kategorien beschrieben.
\\ \\
\textbf{Anmerkung 20 (2.3.5 Pseudo-Code, Seite 5):}
Du schreibst Schritt 2: und Schritt 3:. Für ein einheitliches Schema würde ich Schritt 1: ergänzen.
\\ \\
\textbf{Anmerkung 21 (2.4.1 Definition, Seite 6):}
Ich würde auf das original Paper verweisen. 
Was meint ihr mit Entspannung?
\\ \\
\textbf{Anmerkung 22 (3.1 Bidirektionale Suche, Seite 8):}
\grqq Somit betrachtet jede der beiden Suchen... \glqq{}, Dieser Satz sollte umformuliert werden, weil somit ein Bindewort ist und nicht am Satzanfang stehen sollte. 
\\ \\
\textbf{Anmerkung 23 (3.2 Informed Search, Seite 8):}
Eine Lösung für das Shortest Path Problem wird angegeben, aber das Problem wurde nicht definiert.
\\ \\
\textbf{Anmerkung 24 (3.3 Preprocessing, Seite 9):} 
Dieser Abschnitt sollte umformuliert werden, es wurde \glqq häufig, man, oft\grqq{} verwendet. 
\\ \\
\textbf{Anmerkung 25 (3.3.1 ALT-Algorithmen, Seite 10):}
Der Knoten n und der Zielknoten s wird mal normal und mal kursiv in diesem Abschnitt dargestellt. Hier würde ich n und s einheitlich kursiv darstellen. Außerdem würde ich den letzten Satz wegen \glqq somit\grqq{} umformulieren.
\\ \\
\textbf{Anmerkung 26 (3.3.2 Reach-Based Pruning, Seite 10):}
Außerdem sollte nicht aus Wikipedia zitiert werden, auch nicht in der Fußnote.
\glqq Effizientere Vorgehen wurden in [Gol07] und [Gut04] beschrieben.\grqq, welche effizientere Vorgehensweisen? Vlt. würde ich den Satz einfach entfernen.  
\\ \\
\textbf{Anmerkung 27 (4 Anwendungen, Seite 12):}
Wer hat \glqq In diesem Kapitel wird beschrieben, wie Dijkstra- und Bellman-Algorithmen in realen Anwendungen angewendet werden können.\grqq{} geschrieben? 
\\ \\
\textbf{Anmerkung 28 (4 Anwendungen, Seite 12):}
Wird nur der Algorithmus von Bellman oder der von Bellman-Ford beschrieben?
\\ \\
\textbf{Anmerkung 29 (4.1.1 Definitionen, Seite 12):}
nach Vaibhavi Patel und Prof.Chitra Baggar $->$ Nach Vaibhavi Patel und Prof. Chitra Baggar ist
\\ \\
\textbf{Anmerkung 30 (4.1.1 Definitionen, Seite 12):}
\glqq Obwohl dieser ...\grqq, Dieser Abschnitt liest sich nicht gut. Es werden zu viele Komata verwendet. Diesen würde ich in 3 oder 4 Sätze umformulieren.  
\\ \\
\textbf{Anmerkung 31 (4.1.1 Definitionen, Seite 13):}
\glqq Die Verbesserdungsmethode besteht...\grqq, den Satz würde ich in mehrere Sätze aufteilen. 
\\ \\
\textbf{Anmerkung 32 (4.2.1 Verbesserter Dijkstra, Seite 15):}
\glqq Die Autoren des Artikels haben der verbesserte\grqq $->$ den verbesserten
\\ \\
\textbf{Anmerkung 33 (4.3 Autonome Navigation, Seite 16):}
Du schreibst das die Wahl der richtigen Pfadsuchalgorithmen wichtig ist, aber unter welchen Kriterien wird welcher Algorithmus verwendet? 
Außerdem was ist deine Quelle?
\\ \\
\textbf{Stellungnahme zu 33:}
Sätze wurden umgestellt und die Quelle wurde besser deklariert.
\\ \\
\textbf{Anmerkung 34 (4.3 Autonome Navigation, Seite 16):}
Du schreibst das Forscher Methoden entwickelt haben, um Anforderungen zu erfüllen. Hier könntest du Beispiele nennen was für Methoden entwickelt wurden. 
\\ \\
\textbf{Stellungnahme zu 33:}
Sektion ergänzt um Abschnitt über Methoden/Kriterien nach denen Pfadsuchalgorithmen in der mobilen Navigation ausgewählt werden.
\\ \\
\textbf{Anmerkung 35 (Zusammenfassung):}
Der Begriff iterative Vertiefungssuche taucht das erste Mal in der Zusammenfassung auf. 
\\ \\
\textbf{Stellungnahme zu 33:}
Die iterative Vertiefungssuche wird nun im Kapitel 2.2.2 Tiefensuche als Anwendungsbeispiel für die Tiefensuche erwähnt.
\\ \\
\textbf{Anmerkung 36 (Zusammenfassung):}
Zu welchem Abschnitt wird [RN10f] zugeordnet?
\\ \\
Quellenangaben verbessert und Texte leicht angepasst.
\\ \\
\end{document}