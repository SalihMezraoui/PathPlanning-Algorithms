\chapter{Zusammenfassung und Ausblick}
\label{Zusammenfassung und Ausblick}
In den letzten Jahren hat die Pfadplanung immer mehr an Relevanz gewonnen, zum Beispiel im Bereich des autonomen Autofahrens wird immer mehr Forschung 
im Bereich der Pfadplanungsalgorithmen betrieben. \cite{Karur:21}

In diesem Paper haben wir aufeinander aufbauend Algorithmen vorgestellt, die in einem Programm mit bestimmten Eingaben und Umgebungen 
wie z.B Graphen verwendet werden können um von einem gegeben Start ein oder mehrere Ziele zu finden und dabei durch verschiedene Bewertungskriterien den besten Weg zu finden.
Mit diesen Algorithmen kann ein Programm durch eine Sequenz von Aktionen sein Ziel erreichen. Diesen Prozess nennt man Suche.
\cite{Russell:10c}
\begin{itemize}
    \item Bevor das Programm mit der Suche nach der besten Lösung beginnen kann, muss erst ein Ziel identifiziert und das Problem genau definiert werden.
    \item Suchalgorithmen behandeln Zustände und Aktionen atomar: Sie berücksichtigen keine interne Struktur, die sie besitzen könnten.
    \item Ein allgemeiner uninformierter TREE-SEARCH-Algorithmus berücksichtigt alle möglichen Wege, um eine Lösung zu finden, während ein informierter GRAPH-SEARCH-Algorithmus redundante Wege vermeidet.
    \item Uninformierte Pfadsuchalgorithmen haben nur Zugriff auf die Problemdefinition. Die grundlegenden Algorithmen sind wie folgt:
    \begin{itemize}
        \item Die Breadth-first-Suche expandiert zuerst die flachsten Knoten; sie ist vollständig, optimal für einheitliche Pfadkosten, hat aber eine exponentielle Raumkomplexität.
        \item Die Tiefensuche expandiert zuerst den tiefsten nicht expandierten Knoten. Sie ist weder vollständig noch optimal, hat aber eine lineare Raumkomplexität.
        \item Die iterative Vertiefungssuche ist eine Wiederholung der Tiefensuche mit zunehmender Tiefenbegrenzung, bis ein Ziel gefunden wird. Sie ist vollständig, optimal für die Kosten pro Schritt, hat eine vergleichbare Zeitkomplexität wie die Breitensuche und eine lineare Raumkomplexität.
        \item Der Greedy Dijkstra-Algorithmus der zwar bei der blinden Suche Zeit vergeudet aber dafür optimal ist und eine Trefferquote von 100\% hat.
        \item Die Optimierung durch eine bidirektionale Suche kann die Zeitkomplexität enorm reduzieren, ist aber nicht immer anwendbar und kann zu viel Speicherplatz beanspruchen.
    \end{itemize}
    \item Informierte Suchmethoden können auf heuristische Funktionen zurückgreifen, die die Kosten einer Lösung schätzen
    \begin{itemize}
        \item Der generische Best-First-Suchalgorithmus wählt einen Knoten für die Expansion gemäß einer Bewertungsfunktion aus.
        \item Der Greedy best-first search expandiert Knoten mit minimalem heuristischem Funktionswert. Er ist nicht optimal, aber oft effizient.
        \item A*-Suche expandiert Knoten mit minimalem Heuristischen Funktions- und Pfadkostenwerten. A* ist vollständig und optimal, vorausgesetzt die heuristische Funktion ist zulässig.
        \item ALT-Algorithmen, die aufgebaut auf A* durch Preprocessing noch optimiertere Ergebnisse erzeugt.
        \item Reach-based Pruning, welches den Dijkstra-Algorithmus um eine Metrik erweitert und dadurch optimiert.
    \end{itemize}
\end{itemize}
\cite{Russell:10c}

Alle diese Algorithmen haben diverse Vor- und Nachteile und daraus ergeben sich verschiedene 
Anwendungsbereiche auf welche wir in Kapitel \ref*{Anwendungen} gesondert eingegangen sind.