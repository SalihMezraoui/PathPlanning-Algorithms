\chapter{Anwendungen}

In diesem Kapitel wird beschrieben, warum es unterschiedliche Konsistenzmodelle\index{Konsistenzmodelle} gibt. Außerdem werden die Unterschiede zwischen strengen Konsistenzmodellen\index{Linearisierbarkeit} (Linearisierbarkeit, sequentielle Konsistenz)\index{sequentiell!Konsistenz} und schwachen Konsistenzmodellen\index{Konsistenz!schwach} (schwache Konsistenz, Freigabekonsistenz)\index{Freigabekonsistenz} erläutert. Es wird geklärt, was Strenge und Kosten (billig, teuer) in Zusammenhang mit Konsistenzmodellen bedeuten.

\section{Warum existieren unterschiedliche Konsistenzmodelle?}

Laut \cite{Malte:97} sind mit der\index{Replikation} Replikation von Daten immer zwei gegensätzliche Ziele verbunden: die Erhöhung der\index{Verfügbarkeit} Verfügbarkeit und die Sicherung der\index{Konsistenz} Konsistenz der Daten. Die Form der Konsistenzsicherung bestimmt dabei, inwiefern das eine Kriterium erfüllt und das andere dementsprechend nicht erfüllt ist (Trade-off zwischen Verfügbarkeit und der Konsistenz der Daten). Stark konsistente Daten sind stabil, das heißt, falls mehrere Kopien der Daten existieren, dürfen keine Abweichungen auftreten. Die Verfügbarkeit der Daten ist hier jedoch stark eingeschränkt. Je schwächer die Konsistenz wird, desto mehr Abweichungen können zwischen verschiedenen Kopien einer Datei auftreten, wobei die Konsistenz nur an bestimmten Synchronisationspunkten gewährleistet wird. Dafür steigt aber die Verfügbarkeit der Daten, weil sie sich leichter replizieren lassen.
