
\chapter{Einleitung und Problemstellung}
\label{Einleitung und Problemstellung}


In dieser wissenschaftlichen Arbeit werden aufeinander aufbauend Algorithmen vorgestellt, die in einem Programm mit bestimmten Eingaben und Umgebungen 
wie z.B Graphen verwendet werden können um von einem gegebenen Start ein oder mehrere Ziele zu finden und dabei durch verschiedene Bewertungskriterien den besten Weg zu bestimmen \cite{Esri:00}.
Mit diesen Algorithmen kann ein Programm durch eine Sequenz von Aktionen sein Ziel erreichen. Diesen Prozess nennt man Suche \cite[150-156]{Russell:10}.

Es werden verschiedene Pfadplanungsalgorithmen, wie der Dijkstra-Algorithmus und der A* Algorithmus, 
die im Verkehrleitsystem von Google Maps verwendet werden, präsentiert. \cite{Mehta:19}. 

Das Thema \emph{Funktionsprinzipien und Anwendungen von Algorithmen zur Pfadplanung} hat, damals wie heute einen wichtigen Stellenwert. 
Ob im Bereich der Netzwerkroutenplanung oder bei KI-Spielern in Computerspielen: Pfadsuchalgorithmen sind so relevant wie nie \cite{Foeada:21}. 
Weitere Beispiele für die Anwendung von Pfadsuchalgroithmen sind Planung von öffentlichen Verkehrsmitteln und 
Robotik. Es werden beispielsweise Pakete in Logistikzentren durch Roboter organisiert und Routen über verschiedene Logistikzentren durch Pfadsuchalgorithmen bestimmt.
Viele Bereiche im Alltag verwenden im Hintergrund Pfadsuchalgorithmen um den (kosten)günstigsten Weg zu finden. Dabei ist es wichtig, 
dass diese effizient, akkurat und schnell sein müssen, damit die Hauptsysteme noch genügend Ressourcen übrig haben um wie gewünscht zu funktionieren \cite{Foeada:21}. 

Es werden in dieser wissenschaftlichen Arbeit die Funktionsweise von wichtigen Pfadsuchalgorithmen, angefangen mit den uninformierten Suchalgorithmen wie der Breiten- und Tiefensuche, 
welche einen ersten Einblick in die Thematik geben und die Rahmenbedingungen und Problemumgebungen veranschaulichen sollen. 
Über den Bellman-Ford-Algorithmus, der nahezu intuitiv den kürzesten Weg liefert und bei dem auch negative Kantenlängen möglich sind \cite{Mukhlif:20}. 
Bis hin zu durch Heuristiken optimierte und informierte Pfadsuchalgorithmen wie Dijkstra oder A*, und wie sie in heutiger Zeit 
sonst eingesetzt werden können, veranschaulichen und jeweils (Pseudo-)Code- und Anwendungsbeispiele geben \cite[64]{Russell:10}.
