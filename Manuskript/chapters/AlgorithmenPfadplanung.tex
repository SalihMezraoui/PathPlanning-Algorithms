\chapter{Algorithmen zur Pfadplanung}

In diesem Kapitel wird beschrieben, warum es unterschiedliche Konsistenzmodelle\index{Konsistenzmodelle} gibt. Außerdem werden die Unterschiede zwischen strengen Konsistenzmodellen\index{Linearisierbarkeit} (Linearisierbarkeit, sequentielle Konsistenz)\index{sequentiell!Konsistenz} und schwachen Konsistenzmodellen\index{Konsistenz!schwach} (schwache Konsistenz, Freigabekonsistenz)\index{Freigabekonsistenz} erläutert. Es wird geklärt, was Strenge und Kosten (billig, teuer) in Zusammenhang mit Konsistenzmodellen bedeuten.

\section{Warum existieren unterschiedliche Konsistenzmodelle?}


Nach \cite{Mosberger:93} kann die Performanzsteigerung der schwächeren Konsistenzmodelle wegen der Optimierung\index{Optimierung} (Pufferung, Code-Scheduling, Pipelines) 10-40 Prozent betragen. Wenn man bedenkt, dass mit der Nutzung der vorhandenen Synchronisierungsmechanismen schwächere Konsistenzmodelle den Anforderungen der strengen Konsistenz genügen, stellt sich der höhere programmiertechnischer Aufwand bei der Implementierung der schwächeren Konsistenzmodelle als ihr einziges Manko dar.
