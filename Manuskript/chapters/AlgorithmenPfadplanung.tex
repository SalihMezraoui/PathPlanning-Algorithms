\chapter{Algorithmen zur Pfadplanung}
\section{Was ist Pfadplanung?}
Die Pfadplanung ist ein nichtdeterministisches, polynomialzeitliches ("NP") schweres Problem mit der Aufgabe, 
einen kontinuierlichen Pfad zu finden, der ein System von einer Ausgangs- zu einer Endkonfiguration verbindet. 
Die Komplexität des Problems steigt mit zunehmender Anzahl der Freiheitsgrade des Systems. Der zu verfolgende 
Pfad (der optimale Pfad) wird auf der Grundlage von Einschränkungen und Bedingungen bestimmt, z. B. im Bereich
mobiler Roboter unter Berücksichtigung des kürzesten Weges zwischen den Endpunkten oder der minimalen Fahrzeit
ohne Kollisionen. Manchmal werden Einschränkungen und Ziele gemischt, z. B. um den Energieverbrauch zu minimieren,
ohne dass die Fahrzeit einen bestimmten Schwellenwert überschreitet [1].

\section{Uninformierter Ansatz}
\label{Uninformierter Ansatz}
\textbf{Breitensuche:}\linebreak
Die Breitensuche gehört zu den uninformierten Suchalgorithmen, diese werden auch „blind“ genannt, weil bei ihrer 
Suche auf keine zusätzlichen Informationen (wie z.B. Wichtungen) zurückgegriffen wird.
Bei der Breitensuche wird zunächst vom Wurzelknoten aus betrachtet alle verbunden Knoten ersten Grades besucht
 und dies Ebene für Ebene im Baum wiederholt bis alle Knoten besucht wurden. 
Die Breitensuche findet weitestgehend in der Graphentheorie seine Anwendung.\cite{Russell:10b}
\linebreak\linebreak
\textbf{Tiefensuche:}\linebreak
Die Tiefensuche gehört ebenfalls zu den uninformierten Suchalgorithmen.
Im Gegensatz zu der Breitensuche werden nicht die Ebenen nacheinander abgesucht sondern je Nachfolger angefangen 
beim Wurzelknoten werden bis sie keine weiteren Nachfolger mehr haben besucht. Erst dann wird der nächste Nachbar 
in der ersten Ebene besucht bis keine unbesuchten Knoten mehr vorhanden sind.Die Tiefensuche ist indirekt an vielen 
komplexeren Algorithmen beteiligt. Unter anderem kann die Tiefensuche auch für das Ermitteln von Zusammenhangskomponenten
oder für das Erzeugen eines Irrgartens verwendet werden.
\cite{Russell:10c}